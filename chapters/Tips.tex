\chapter{Troubleshooting and Adjustments} \label{ch:Tips}

This chapter summarises common configuration and troubleshooting scenarios reported for Digifiz Replica Next and classic Digifiz Replica dashboards.
Each entry lists the recommended diagnostic steps and the configuration commands involved.

\section{Replica Next}

\begin{description}
    \item[Hotspot not visible] Move closer to the dashboard and ensure the vehicle is in an open area. Disable cellular data and reconnect to \texttt{Digifiz\_AP} (or \texttt{PHOL-LABS2}).
    \item[404 when opening \texttt{192.168.4.1}] Confirm that mobile data is disabled and retry the connection.
    \item[Firmware updates] Navigate to the \emph{WiFi} tab, select the provided \texttt{Digifiz.bin} release (latest builds are published at \url{https://github.com/Sgw32/DigifizReplica/releases}), click \emph{Upload}, and repeat if the upload fails. Record the current odometer value and restore it with command \verb|11 <mileage>| after the update if necessary.
    \item[Commands ignored] Refresh the web page, reopen the \emph{Control} tab, and send the command again.
    \item[Speed reading incorrect] Connect via Wi-Fi, drive at an indicated 100~km/h, note the GPS speed, then issue \verb|1 <gps_value>| (for example, \verb|1 85|) to adjust \texttt{PARAMETER\_SPEEDCOEFFICIENT}.
    \item[RPM reading incorrect] Issue \verb|22 <value>| (for example, \verb|22 1500| to halve the reading) to recalibrate \texttt{PARAMETER\_RPMCOEFFICIENT}. Audi clusters typically use \verb|22 3000|.
    \item[Display too dim] Disable automatic brightness with \verb|13 0| and increase manual brightness (for example, \verb|14 50|). Use conservative values (below 60) to extend LED life.
    \item[Setting the clock] From the web terminal (or Serial Bluetooth Terminal on legacy builds), enter \verb|255 <hours>| followed by \verb|254 <minutes>| (for example, \verb|255 23| and \verb|254 55|).
    \item[Fuel level stuck at zero or maximum] Disconnect the battery, verify that the sender resistance measured between the harness pin and vehicle ground is within 30--300~\ohm{}, and inspect the connector for continuity. Short circuits (<5~\ohm{}) must be eliminated. If the resistance varies correctly but the gauge does not, capture \verb|adc 0| readings at multiple fuel levels and share them with the authors.
    \item[Fuel flow readings inaccurate] The optional flow sensor delivers emulated data and is unreliable without the manifold pressure sensor; treat the readings as experimental.
    \item[Coolant temperature scaling] Adjust \texttt{PARAMETER\_COOLANT\_MIN\_R} and \texttt{PARAMETER\_COOLANT\_MAX\_R} to tune the display range (for example, \verb|27 30| lowers the ``1~bar'' threshold to 30~\,^{\circ}C).
    \item[Oil or ambient temperature missing] Disconnect the battery and measure the sensor resistance: oil sensors should read about 2~k\ohm{}, ambient sensors about 10~k\ohm{} at room temperature. If necessary, adjust \texttt{PARAMETER\_NORMAL\_RESISTANCE\_OIL} (command 20) or \texttt{PARAMETER\_NORMAL\_RESISTANCE\_AMB} (command 21); values below the default lower the indicated temperature, higher values raise it. Persistent issues should be escalated with \verb|adc 0| readings.
    \item[Changing UI colours] Use commands 31--33 to set the RGB components or update to the latest firmware to access the visual colour controls in the web UI.
\end{description}

\section{Classic Digifiz Replica}

\begin{description}
    \item[Bluetooth module not visible] Confirm that the host uses Bluetooth Classic rather than BLE. iOS devices cannot connect to the Bluetooth~2.0 module.
    \item[Commands rejected] Firmware builds from 2024 and newer require unlocking with \verb|234 123| before accepting configuration commands.
    \item[Speed or RPM calibration] Adjust \texttt{PARAMETER\_SPEEDCOEFFICIENT} with \verb|1 <gps_value>| and \texttt{PARAMETER\_RPMCOEFFICIENT} with \verb|0 <value>| or \verb|22 <value>|, using the same approach as Replica Next.
    \item[Manual brightness adjustment] Disable automatic brightness with \verb|13 0| and set \verb|14 15| for maximum brightness. Re-enable automatic control with \verb|13 1| if required.
    \item[Clock adjustment] Enter \verb|255 <hours>| followed by \verb|254 <minutes>| through Serial Bluetooth Terminal.
    \item[Fuel gauge issues] With the battery disconnected, verify the sender resistance on the harness (30--300~\ohm{}) and inspect for shorts. Check the signal path to the main board and clean contacts if needed.
    \item[Fuel flow accuracy] The optional flow sensor relies on intake manifold pressure data and otherwise produces approximate readings.
    \item[Coolant temperature calibration] Tune \texttt{PARAMETER\_COOLANT\_MIN\_R} and \texttt{PARAMETER\_COOLANT\_MAX\_R} experimentally to suit the vehicle (for example, \verb|27 30|).
\end{description}

For additional parameter defaults and configuration details, refer to \Cref{tbl:next-commands,tbl:next-defaults} and the classic Replica command list in \Cref{appendix:reference}.
