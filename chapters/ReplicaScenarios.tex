\chapter{Typical situations for setting up the Digifiz Replica}\label{ch:replica-scenarios}

Before troubleshooting, confirm that the dashboard is the classic Digifiz Replica (\autoref{ch:replica-setup}). Replica Next panels use a Wi-Fi portal and are covered in \autoref{ch:replica-next-scenarios}.

\begin{description}
    \item[Bluetooth module not detected] Pair with the dashboard's Bluetooth Classic module. Serial Bluetooth Terminal for Android is the recommended tool; BLE-only scanners will not find the device.
    \item[iPhone or iPad cannot connect] Classic Replica dashboards use Bluetooth~2.0 and are incompatible with iOS devices. Use an Android phone or a computer with a Bluetooth serial utility.
    \item[Commands ignored on 2024+ firmware] Unlock the protocol by sending \verb|234 123|, then resend the required command sequence.
    \item[Speed reading too high or low] Connect via Serial Bluetooth Terminal, drive at an indicated \SI{100}{\kilo\metre\per\hour}, and note the GPS speed. Send \verb|1 <gps_value>| (for example, \verb|1 85|) to set \texttt{PARAMETER\_SPEEDCOEFFICIENT} to the GPS value.
    \item[RPM reading incorrect] Older firmware uses \verb|0 <value>| while current builds use \verb|22 <value>|. Audi engines typically require \verb|22 3000|; halve or double the value to correct the display (for example, \verb|22 1500|).
    \item[Increase brightness] Disable automatic control with \verb|13 0|, then set \verb|14 15| (or another preferred value) for maximum manual brightness. Re-enable automatic control later with \verb|13 1|.
    \item[Setting the clock] Send \verb|255 <hours>| followed by \verb|254 <minutes>|. Examples: \verb|255 23| and \verb|254 55| for 23:55; \verb|255 14| and \verb|254 30| for 14:30; \verb|255 2| and \verb|254 28| for 02:28.
    \item[Fuel gauge issues] Disconnect the vehicle battery before measuring.\begin{itemize}
        \item If the display drops slowly from 60 to 0, measure the sender resistance between the harness pin and ground; valid readings are typically \SIrange{30}{300}{\ohm}. Check the connector and the feed into the main board.
        \item If the gauge is pegged full, look for a short to ground (less than \SI{5}{\ohm}) on the sender line and repair it.
        \item If the reading never changes, compare the sender resistance with full and empty tanks. Replace the sensor if the resistance is constant.
    \end{itemize}
    \item[Fuel flow values seem wrong] The displayed flow is emulated unless an intake-manifold pressure sensor is installed. Treat the reading as indicative only.
    \item[Coolant gauge inaccurate] Adjust \texttt{PARAMETER\_COOLANT\_MIN\_R} and \texttt{PARAMETER\_COOLANT\_MAX\_R}. Example: \verb|27 30| shortens the scale so that ``1~bar'' corresponds to roughly \SI{30}{\celsius}.
\end{description}

Collect raw sensor data with \verb|adc 0| if the problem persists and share the results with the dashboard developers for analysis.
