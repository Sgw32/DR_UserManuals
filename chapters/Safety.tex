\chapter{Operating conditions and safety precautions}\label{ch:safety}

\section{Environmental limits}
\begin{itemize}
    \item The instrument panel operates between \(-40\,^{\circ}\mathrm{C}\) and \(+70\,^{\circ}\mathrm{C}\) at relative humidity up to 95~\%.
    \item The dashboard may remain installed inside the vehicle throughout the year, including when the car is parked for extended periods.
\end{itemize}

\section{Safety precautions}
\begin{enumerate}
    \item The Digifiz dashboard is a do-it-yourself device assembled and integrated by enthusiasts. Observe general electrical safety practices while working with it.
    \item The product is intended for the personal projects of vehicle owners.
    \item The readings are not certified or metrologically verified, although they correspond to the declared specifications at the time of release.
    \item Use the dashboard only when you accept responsibility for the installation and for road safety.
    \item If the displayed data cannot be trusted, verify it with the vehicle's standard gauges or external measuring instruments.
    \item Do not use the instrument panel outputs for automatic vehicle control systems.
    \item The authors accept no liability for consequences arising from the installation or use of the dashboard, including traffic fines or accidents. Malfunctions reported within the warranty period (one year for installations performed jointly with the authors and two weeks for independent installations) will be repaired.
    \item The functional capabilities listed in \Cref{ch:technical-specs} are guaranteed for one year during supervised installation and for two weeks after independent installation.
\end{enumerate}
