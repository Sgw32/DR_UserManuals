\chapternotnumbered{Introduction} \label{ch:Introduction}

This user manual consolidates the build notes, service procedures, and release checklists gathered while shipping Digifiz Replica dashboards across two hardware generations.
The \TeXtured{} template powering the document was iterated on for years to balance production-ready typography with a modular source layout that stays maintainable as the manual grows.

Refining the template for technical documentation meant embracing theorem-like and remark-like environments for tightly connected references, and investing into automation so that metadata, figures, and indexes stay synchronized between revisions.
Improved understanding of the coding backbone behind \LaTeX{} and its package ecosystem enabled me to customize it even further to my liking, and add even more \enquote{bells and whistles}.

\begin{remark}[Template Purpose]
    While this project focuses on the Digifiz Replica user manual, \TeXtured{} can be used for other long-form technical documents as well.
\end{remark}

To make it user-friendly, I have restructured the preamble into several files, each of which is responsible for a specific aspect of the document.
This way, the user can (and is encouraged to) easily find the relevant part of the code and modify it.

Numerous comments and explanations are provided throughout the code to further aid the user in understanding the template without always having to consult the documentation of packages (which is recommended for more advanced changes).

\begin{remark}[How to Setup]
    To set up \TeXtured{} template for your document, you can use the \texttt{Overleaf} template or clone the repository on \textsf{GitHub} \autocite{TeXtured}.
    Then, you can start modifying the files to suit your needs.

    Also make sure to check the \texttt{README.md} file for more detailed instructions, particularly on various software dependencies.
    If you encounter any issues, please see \href{https://github.com/jdujava/TeXtured/issues/2}{\texttt{jdujava/TeXtured \#2}} and \href{https://github.com/jdujava/TeXtured/issues/5}{\texttt{jdujava/TeXtured \#5}}.
\end{remark}
