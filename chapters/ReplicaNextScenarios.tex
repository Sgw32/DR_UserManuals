\chapter{Typical situations for setting up the \ReplicaNextLong{}}\label{ch:replica-next-scenarios}

\begin{description}
    \item[Hotspot not visible] Move closer to the vehicle and ensure it is parked in an open area. Disable mobile data, forget stale Wi-Fi profiles, and reconnect to \texttt{Digifiz\_AP} (or \texttt{PHOL-LABS2}).
    \item[404 at \texttt{192.168.4.1}] Turn off mobile data on the phone or laptop and reload the page. Captive portal detection on Android/iOS often interferes until the cellular modem is disabled.
    \item[Firmware updates] Open the \emph{WiFi} tab, select the supplied \texttt{Digifiz.bin} file (the latest releases are published at \url{https://github.com/Sgw32/DigifizReplica/releases}), and click \emph{Upload}. The first attempt can fail; repeat the upload if necessary. Successful flashes redirect to a confirmation page. Record the odometer before updating and restore it afterwards with command \InlineCode{11} and the stored mileage value.
    \item[Commands ignored] Refresh the browser, return to the \emph{Control} tab, and resend the command. Ensure the \emph{Process} button is pressed after entering the value.
    \item[Speed reading incorrect] Connect via Wi-Fi, drive at an indicated \SI{100}{\kilo\metre\per\hour}, note the GPS speed, then issue command \InlineCode{1} with the measured value (for example, \InlineCode{1 85}) to set \texttt{PARAMETER\_SPEEDCOEFFICIENT} accordingly.
    \item[RPM reading incorrect] Adjust \texttt{PARAMETER\_RPMCOEFFICIENT}. Older firmware uses command \InlineCode{0} followed by the value; current versions use \InlineCode{22}. Example: \InlineCode{22 1500} halves the reading relative to \InlineCode{22 3000}.
    \item[Display too dim] Disable automatic brightness with \InlineCode{13 0}, then raise the manual level (for example, \InlineCode{14 50}). Experiment with values between 45 and 55; avoid levels above 60 to preserve LED life.
    \item[Setting the clock] Use the web terminal (or Serial Bluetooth Terminal on legacy builds) to send command \InlineCode{255} with the hour value followed by \InlineCode{254} with the minutes. Example: \InlineCode{255 23} and \InlineCode{254 55} sets 23:55.
    \item[Fuel readings stuck] Disconnect the battery and measure the resistance between the fuel sender pin and vehicle ground. Valid readings are typically \SIrange{30}{300}{\ohm}. Repair shorts below \SI{5}{\ohm} or open circuits before reconnecting. If the readings vary correctly but the gauge does not, record \InlineCode{adc 0} results at several fuel levels and share them with PHOL-LABS Kft.
    \item[Fuel flow readings inaccurate] The optional flow sensor produces emulated data and is unreliable without an intake manifold pressure sensor. Treat the readings as experimental.
    \item[Coolant temperature out of range] Tune \texttt{PARAMETER\_COOLANT\_MIN\_R} and \texttt{PARAMETER\_COOLANT\_MAX\_R}. Example: \InlineCode{27 30} lowers the ``1~bar'' threshold to \SI{30}{\celsius}.
    \item[Oil or ambient temperature missing] With the battery disconnected and the engine cold, measure the sensor resistance. Oil sensors should read about \SI{2}{\kilo\ohm} \ensuremath{\pm}\SI{0.3}{\kilo\ohm}, ambient sensors about \SI{10}{\kilo\ohm} \ensuremath{\pm}\SI{2}{\kilo\ohm}. Adjust \texttt{PARAMETER\_NORMAL\_RESISTANCE\_OIL} (command~20) or \texttt{PARAMETER\_NORMAL\_RESISTANCE\_AMB} (command~21); lower values decrease the indicated temperature, higher values increase it. Persistent issues should be diagnosed by collecting \InlineCode{adc 0} output and contacting PHOL-LABS Kft.
    \item[Changing interface colour] Use commands 31--33 to set the RGB values. New firmware revisions include visual colour controls in the web interface, so update regularly.
\end{description}
