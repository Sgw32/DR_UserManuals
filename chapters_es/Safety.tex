\chapter{Condiciones de funcionamiento y precauciones de seguridad}\label{ch:safety}

\section{Límites ambientales}
\begin{itemize}
    \item El cuadro de instrumentos funciona entre \(-40\,^{\circ}\mathrm{C}\) y \(+70\,^{\circ}\mathrm{C}\) con una humedad relativa de hasta el 95~\%.
    \item El tablero puede permanecer instalado en el vehículo durante todo el año, incluso cuando el automóvil permanece estacionado por periodos prolongados.
\end{itemize}

\section{Precauciones de seguridad}
\begin{enumerate}
    \item El cuadro Digifiz es un dispositivo de bricolaje ensamblado e integrado por entusiastas. Observe las prácticas generales de seguridad eléctrica al trabajar con él.
    \item El producto está destinado a los proyectos personales de los propietarios de vehículos.
    \item Las lecturas no están certificadas ni verificadas metrológicamente, aunque corresponden a las especificaciones declaradas en el momento del lanzamiento.
    \item Utilice el tablero únicamente cuando acepte la responsabilidad de la instalación y de la seguridad vial.
    \item Si no confía en los datos mostrados, verifíquelos con los indicadores estándar del vehículo o con instrumentos de medición externos.
    \item No utilice las salidas del cuadro de instrumentos para sistemas de control automático del vehículo.
    \item Los autores no aceptan responsabilidad por las consecuencias derivadas de la instalación o el uso del tablero, incluidas multas de tráfico o accidentes. Las averías comunicadas dentro del periodo de garantía (un año para instalaciones realizadas conjuntamente con los autores y dos semanas para instalaciones independientes) serán reparadas.
    \item Las capacidades funcionales enumeradas en \Cref{ch:technical-specs} están garantizadas durante un año en instalaciones supervisadas y durante dos semanas después de una instalación independiente.
\end{enumerate}
