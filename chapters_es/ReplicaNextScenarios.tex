\chapter{Situaciones típicas al configurar \ReplicaNextShort{}}\label{ch:replica-next-scenarios}

\begin{description}
    \item[Punto de acceso no visible] Acérquese al vehículo y asegúrese de que esté estacionado en un área abierta. Desactive los datos móviles, elimine perfiles Wi-Fi obsoletos y vuelva a conectarse a \texttt{Digifiz\_AP} (o \texttt{PHOL-LABS2}).
    \item[404 en \texttt{192.168.4.1}] Apague los datos móviles del teléfono o portátil y recargue la página. La detección del portal cautivo en Android/iOS suele interferir hasta que se desactiva el módem celular.
    \item[Actualizaciones de firmware] Abra la pestaña \emph{WiFi} y seleccione el archivo \texttt{Digifiz.bin} suministrado. Las últimas versiones se publican en el enlace siguiente.
        \displayurl{https://github.com/Sgw32/DigifizReplica/releases}
        Haga clic en \emph{Upload}. El primer intento puede fallar; repita la carga si es necesario. Una escritura correcta redirige a una página de confirmación. Registre el odómetro antes de actualizar y restáurelo después con \verb|11 <mileage>|.
    \item[Comandos ignorados] Actualice el navegador, regrese a la pestaña \emph{Control} y reenvíe el comando. Asegúrese de pulsar el botón \emph{Process} tras introducir el valor.
    \item[Lectura de velocidad incorrecta] Conéctese por Wi-Fi, conduzca a \SI{100}{\kilo\metre\per\hour} indicados, anote la velocidad GPS y ejecute \verb|1 <gps_value>| (por ejemplo, \verb|1 85|) para establecer \paramname{PARAMETER\_SPEEDCOEFFICIENT} en el valor verificado.
    \item[Lectura de RPM incorrecta] Ajuste \paramname{PARAMETER\_RPMCOEFFICIENT}. El firmware antiguo usa \verb|0 <valor>|; las versiones actuales usan \verb|22 <valor>|. Ejemplo: \verb|22 1500| reduce a la mitad la lectura respecto a \verb|22 3000|.
    \item[Pantalla demasiado tenue] Desactive el brillo automático con \verb|13 0| y aumente el nivel manual (por ejemplo, \verb|14 50|). Pruebe valores entre 45 y 55; evite niveles superiores a 60 para preservar la vida útil de los LED.
    \item[Ajustar el reloj] Utilice el terminal web (o Serial Bluetooth Terminal en compilaciones antiguas) para enviar \verb|255 <hours>| seguido de \verb|254 <minutes>|. Ejemplo: \verb|255 23| y \verb|254 55| establece 23:55.
    \item[Lecturas de combustible congeladas] Desconecte la batería y mida la resistencia entre el pin del aforador y la masa del vehículo. Las lecturas válidas suelen estar entre \SIrange{30}{300}{\ohm}. Repárese los cortocircuitos por debajo de \SI{5}{\ohm} o los circuitos abiertos antes de reconectar. Si las lecturas varían correctamente pero el indicador no, registre los resultados de \verb|adc 0| en varios niveles de combustible y compártalos con PHOL-LABS Kft.
    \item[Lecturas de flujo de combustible imprecisas] El sensor de flujo opcional produce datos emulados y es poco fiable sin un sensor de presión del colector de admisión. Considere las lecturas como experimentales.
    \item[Temperatura del refrigerante fuera de rango] Ajuste \paramname{PARAMETER\_COOLANT\_MIN\_R} y \paramname{PARAMETER\_COOLANT\_MAX\_R}. Ejemplo: \verb|27 30| reduce el umbral de “1~bar” a \SI{30}{\celsius}.
    \item[Falta la temperatura de aceite o ambiente] Con la batería desconectada y el motor frío, mida la resistencia del sensor. Los sensores de aceite deberían marcar alrededor de \SI{2}{\kilo\ohm} \ensuremath{\pm}\SI{0.3}{\kilo\ohm}; los sensores de ambiente, alrededor de \SI{10}{\kilo\ohm} \ensuremath{\pm}\SI{2}{\kilo\ohm}. Ajuste \paramname{PARAMETER\_NORMAL\_RESISTANCE\_OIL} (comando~20) o \paramname{PARAMETER\_NORMAL\_RESISTANCE\_AMB} (comando~21); valores menores disminuyen la temperatura indicada, valores mayores la incrementan. Los problemas persistentes deben diagnosticarse recopilando la salida de \verb|adc 0| y contactando con PHOL-LABS Kft.
    \item[Cambio de color de la interfaz] Utilice los comandos 31--33 para fijar los valores RGB. Las nuevas versiones de firmware incorporan controles visuales de color en la interfaz web, por lo que conviene actualizar con regularidad.
\end{description}
