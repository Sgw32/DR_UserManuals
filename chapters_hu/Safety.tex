\chapter{Üzemi feltételek és biztonsági előírások}\label{ch:safety}

\section{Környezeti határértékek}
\begin{itemize}
    \item A műszeregység \(-40\,^{\circ}\mathrm{C}\) és \(+70\,^{\circ}\mathrm{C}\) között működik, legfeljebb 95~\% relatív páratartalom mellett.
    \item A műszer egész évben a járműben maradhat, még akkor is, ha az autó huzamosabb ideig áll.
\end{itemize}

\section{Biztonsági előírások}
\begin{enumerate}
    \item A Digifiz műszerfal barkács jellegű, rajongók által összeszerelt és beépített eszköz. A munkavégzés során tartsa be az általános villamos biztonsági szabályokat.
    \item A termék járműtulajdonosok saját projektjeihez készült.
    \item A kijelzések nem minősített vagy metrológiailag hitelesített adatok, bár a kiadás időpontjában megfelelnek a megadott specifikációknak.
    \item Csak akkor használja a műszeregységet, ha vállalja a beépítésért és a közlekedésbiztonságért a felelősséget.
    \item Ha a kijelzett értékek megbízhatósága kérdéses, ellenőrizze azokat a jármű gyári műszereivel vagy külső mérőeszközökkel.
    \item Ne használja a műszeregység kimeneteit automatikus járművezérlő rendszerekhez.
    \item A szerzők nem vállalnak felelősséget a műszerfal beépítéséből vagy használatából fakadó következményekért, beleértve a közlekedési bírságokat vagy baleseteket. A garanciaidőn belül jelentett hibákat kijavítjuk (egy év, ha a beépítés a szerzők felügyeletével történt, illetve két hét önálló beépítés esetén).
    \item A \Cref{ch:technical-specs} fejezetben felsorolt funkcionális képességek egy évig garantáltak a felügyelt beépítést követően, illetve két hétig önálló beépítés után.
\end{enumerate}
