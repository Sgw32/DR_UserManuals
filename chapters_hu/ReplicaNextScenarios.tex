\chapter{Tipikus beállítási helyzetek \ReplicaNextShort{} műszerekhez}\label{ch:replica-next-scenarios}

\begin{description}
    \item[A hotspot nem látható] Menjen közelebb a járműhöz, és győződjön meg róla, hogy nyílt területen áll. Kapcsolja ki a mobiladatot, felejtse el a régi Wi-Fi profilokat, majd csatlakozzon újra a \texttt{Digifiz\_AP} (vagy \texttt{PHOL-LABS2}) hálózathoz.
    \item[404-es hiba a \texttt{192.168.4.1} címen] Kapcsolja ki a mobiladatot a telefonon vagy laptopon, és frissítse az oldalt. Az Android/iOS captive portal érzékelése gyakran zavar, amíg a mobilmodem aktív.
    \item[Firmware frissítés] Nyissa meg a \emph{WiFi} lapot, és válassza ki a mellékelt \texttt{Digifiz.bin} fájlt. A legújabb kiadások az alábbi címen érhetők el.
        \displayurl{https://github.com/Sgw32/DigifizReplica/releases}
        Kattintson az \emph{Upload} gombra. Az első kísérlet sikertelen lehet; szükség esetén ismételje meg. Sikeres feltöltés után a portál visszaigazoló oldalra irányít. A frissítés előtt jegyezze fel a kilométer-számlálót, majd állítsa vissza \verb|11 <futásteljesítmény>| paranccsal.
    \item[Figyelmen kívül hagyott parancsok] Frissítse a böngészőt, lépjen vissza a \emph{Control} lapra, és küldje el újra a parancsot. Győződjön meg róla, hogy az érték beírása után megnyomta a \emph{Process} gombot.
    \item[Téves sebességkijelzés] Wi-Fi kapcsolaton keresztül, menet közben \SI{100}{\kilo\metre\per\hour} kijelzett sebességnél jegyezze fel a GPS értéket, majd adja ki a \verb|1 <gps_érték>| parancsot (például \verb|1 85|), így a \paramname{PARAMETER\_SPEEDCOEFFICIENT} a hiteles értékre áll.
    \item[Téves fordulatszám] Állítsa be a \paramname{PARAMETER\_RPMCOEFFICIENT} értéket. A régebbi firmware \verb|0 <érték>| formát használ, az aktuális verziók \verb|22 <érték>| parancsot várnak. Példa: a \verb|22 1500| a \verb|22 3000| értékhez képest felezi a kijelzést.
    \item[Túl sötét kijelző] Tiltsa le az automatikus fényerőt a \verb|13 0| paranccsal, majd emelje a kézi szintet (például \verb|14 50|). 45 és 55 közötti értékekkel kísérletezzen; 60 fölé ne menjen, hogy megőrizze a LED-ek élettartamát.
    \item[Óra beállítása] Küldje el a webes terminálon (vagy régi kiadásokon a Serial Bluetooth Terminalban) a \verb|255 <óra>| parancsot, majd a \verb|254 <perc>| parancsot. Példa: \verb|255 23| és \verb|254 55| beállítja a 23:55-öt.
    \item[Beragadt üzemanyagszint] Húzza le az akkumulátort, és mérje meg az üzemanyagjeladó csap és a jármű testpontja közötti ellenállást. A tipikus érték \SIrange{30}{300}{\ohm}. Javítsa a \SI{5}{\ohm} alatti rövidzárakat vagy a szakadásokat, mielőtt visszakötné. Ha a mérés változik, de a műszer nem reagál, rögzítsen \verb|adc 0| eredményeket több üzemanyagszintnél, és ossza meg a PHOL-LABS Kft.-vel.
    \item[Üzemanyagáramlás pontatlan] Az opcionális átfolyásmérő emulált adatokat ad, és szívócső-nyomásérzékelő nélkül megbízhatatlan. Kezelje kísérleti funkcióként.
    \item[Kívül eső hűtőfolyadék-hőmérséklet] Finomhangolja a \paramname{PARAMETER\_COOLANT\_MIN\_R} és \paramname{PARAMETER\_COOLANT\_MAX\_R} értékeket. Példa: a \verb|27 30| a „1~bar” küszöböt \SI{30}{\celsius}-ra csökkenti.
    \item[Hiányzó olaj- vagy külső hőmérséklet] Leválasztott akkumulátorral és hideg motorral mérje meg az érzékelő ellenállását. Az olajszenzorok jellemzően \ensuremath{2\pm0.3}~\si{k\ohm}, a külső hőmérséklet érzékelők \ensuremath{10\pm2}~\si{k\ohm}. Állítsa a \paramname{PARAMETER\_NORMAL\_RESISTANCE\_OIL} (20-as parancs) vagy a \paramname{PARAMETER\_NORMAL\_RESISTANCE\_AMB} (21-es parancs) értékét; az alacsonyabb szám csökkenti, a magasabb növeli a kijelzett hőmérsékletet. Tartós hibák esetén gyűjtsön \verb|adc 0| adatokat, és vegye fel a kapcsolatot a PHOL-LABS Kft.-vel.
    \item[Felület színének módosítása] A 31--33 parancsokkal állítsa az RGB értékeket. Az új firmware kiadások vizuális színszabályzókat tartalmaznak a webes felületen, ezért érdemes rendszeresen frissíteni.
\end{description}
