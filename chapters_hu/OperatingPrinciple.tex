\chapter{Működési elv}\label{ch:operating-principle}

A \ReplicaGenOne{} műszeregységek az eredeti Volkswagen házat, a gyári CE~1 vagy CE~2 csatlakozókat, valamint a mechanikus sebességmérő bowdent vagy egy elektronikus sebességjeladót használnak.
A \ReplicaGenOneShort{} főpanelje üvegszálas NYÁK-ra épül, diszkrét alkatrészekkel, amelyeket egy ATmega~2560 mikrokontroller és MAX~7219 kijelzőmeghajtók vezérelnek.

A \ReplicaNextLong{} alapját egy ESP32-S3 rendszerchip adja; újonnan gyártott SLA-nyomtatott házat, áttervezett előlapot és takarólemezt, valamint egy csatlakozó-adapterpanelt kapott.
A \ReplicaNextShort{} kijelzőjét WS2812 címzett LED-ek világítják meg az előlapi keret mögül, a hozzá tartozó kábelkorbács pedig alapértelmezetten tartalmazza az elektronikus sebességérzékelőt.

Mindkét generáció azonos kijelző-elrendezést és MFA oldalakat kínál, így a beszerelési folyamat és a mindennapi használat a hardververziók között is ismerős marad.
