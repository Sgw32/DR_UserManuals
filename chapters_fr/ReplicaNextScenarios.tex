\chapter{Situations typiques lors du réglage de \ReplicaNextShort{}}\label{ch:replica-next-scenarios}

\begin{description}
    \item[Hotspot invisible] Approchez-vous du véhicule et assurez-vous qu'il soit stationné dans une zone dégagée. Coupez les données mobiles, oubliez les profils Wi-Fi obsolètes et reconnectez-vous à \texttt{Digifiz\_AP} (ou \texttt{PHOL-LABS2}).
    \item[Erreur 404 sur \texttt{192.168.4.1}] Désactivez les données mobiles du téléphone ou de l'ordinateur et rechargez la page. La détection de portail captif d'Android/iOS interfère souvent tant que le modem cellulaire reste actif.
    \item[Mise à jour de micrologiciel] Ouvrez l'onglet \emph{WiFi} et sélectionnez le fichier \texttt{Digifiz.bin} fourni. Les dernières versions sont publiées au lien ci-dessous.
        \displayurl{https://github.com/Sgw32/DigifizReplica/releases}
        Cliquez sur \emph{Upload}. La première tentative peut échouer ; relancez le transfert si besoin. Un flash réussi redirige vers une page de confirmation. Notez l'odomètre avant la mise à jour et restaurez-le ensuite avec \verb|11 <kilométrage>|.
    \item[Commandes ignorées] Actualisez le navigateur, revenez à l'onglet \emph{Control} et renvoyez la commande. Vérifiez que le bouton \emph{Process} soit pressé après la saisie.
    \item[Vitesse erronée] Connectez-vous en Wi-Fi, roulez à \SI{100}{\kilo\metre\per\hour} indiqués, relevez la vitesse GPS puis envoyez \verb|1 <valeur_gps>| (par exemple \verb|1 85|) pour régler \paramname{PARAMETER\_SPEEDCOEFFICIENT} sur la valeur vérifiée.
    \item[Régime erroné] Ajustez \paramname{PARAMETER\_RPMCOEFFICIENT}. Les anciens micrologiciels utilisent \verb|0 <valeur>| ; les versions actuelles \verb|22 <valeur>|. Exemple~: \verb|22 1500| divise l'affichage par deux par rapport à \verb|22 3000|.
    \item[Affichage trop sombre] Désactivez la luminosité automatique avec \verb|13 0|, puis augmentez le niveau manuel (par exemple \verb|14 50|). Testez des valeurs entre 45 et 55 ; évitez de dépasser 60 pour préserver les LED.
    \item[Réglage de l'horloge] Utilisez le terminal web (ou Serial Bluetooth Terminal sur les anciens modèles) pour envoyer \verb|255 <heures>| suivi de \verb|254 <minutes>|. Exemple~: \verb|255 23| et \verb|254 55| règlent 23:55.
    \item[Jauge de carburant figée] Débranchez la batterie et mesurez la résistance entre la broche de jauge et la masse véhicule. Les valeurs valides se situent généralement entre \SIrange{30}{300}{\ohm}. Corrigez les courts-circuits sous \SI{5}{\ohm} ou les circuits ouverts avant de reconnecter. Si les mesures varient correctement mais pas l'afficheur, relevez les résultats de \verb|adc 0| à plusieurs niveaux et partagez-les avec PHOL-LABS Kft.
    \item[Débit de carburant imprécis] Le capteur optionnel génère des données simulées et reste peu fiable sans capteur de pression de collecteur d'admission. Considérez ces valeurs comme expérimentales.
    \item[Température liquide hors plage] Réglez \paramname{PARAMETER\_COOLANT\_MIN\_R} et \paramname{PARAMETER\_COOLANT\_MAX\_R}. Exemple~: \verb|27 30| abaisse le seuil « 1~bar » à \SI{30}{\celsius}.
    \item[Température huile ou ambiante absente] Batterie débranchée et moteur froid, mesurez la résistance de la sonde. Les sondes d'huile doivent afficher environ \SI{2}{\kilo\ohm} \ensuremath{\pm}\SI{0.3}{\kilo\ohm}, les sondes ambiantes environ \SI{10}{\kilo\ohm} \ensuremath{\pm}\SI{2}{\kilo\ohm}. Ajustez \paramname{PARAMETER\_NORMAL\_RESISTANCE\_OIL} (commande~20) ou \paramname{PARAMETER\_NORMAL\_RESISTANCE\_AMB} (commande~21) ; des valeurs plus basses diminuent la température indiquée, des valeurs plus hautes l'augmentent. En cas de problème persistant, collectez la sortie de \verb|adc 0| et contactez PHOL-LABS Kft.
    \item[Changer la couleur de l'interface] Utilisez les commandes 31--33 pour définir les composantes RVB. Les nouvelles versions du micrologiciel proposent des réglages visuels dans le portail web ; mettez-le à jour régulièrement.
\end{description}
