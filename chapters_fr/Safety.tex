\chapter{Conditions d'utilisation et consignes de sécurité}\label{ch:safety}

\section{Limites environnementales}
\begin{itemize}
    \item Le combiné fonctionne entre \(-40\,^{\circ}\mathrm{C}\) et \(+70\,^{\circ}\mathrm{C}\) pour une humidité relative pouvant atteindre 95~\%.
    \item Le tableau de bord peut rester installé dans le véhicule toute l'année, y compris lors d'un stationnement prolongé.
\end{itemize}

\section{Consignes de sécurité}
\begin{enumerate}
    \item Le tableau Digifiz est un appareil assemblé et intégré par des passionnés. Respectez les règles générales de sécurité électrique lors des travaux.
    \item Le produit est destiné aux projets personnels des propriétaires de véhicules.
    \item Les indications ne sont ni certifiées ni vérifiées métrologiquement, bien qu'elles correspondent aux spécifications annoncées lors de la publication.
    \item N'utilisez le tableau de bord que si vous acceptez la responsabilité de l'installation et de la sécurité routière.
    \item Si les données affichées vous semblent douteuses, vérifiez-les à l'aide des instruments standard du véhicule ou d'appareils de mesure externes.
    \item N'utilisez pas les sorties du combiné pour des systèmes de contrôle automatique du véhicule.
    \item Les auteurs déclinent toute responsabilité quant aux conséquences liées à l'installation ou à l'utilisation du tableau, y compris les amendes ou accidents. Les dysfonctionnements signalés durant la période de garantie (un an pour les installations réalisées avec les auteurs et deux semaines pour les installations autonomes) seront réparés.
    \item Les performances fonctionnelles décrites au \Cref{ch:technical-specs} sont garanties pendant un an dans le cadre d'une installation supervisée et durant deux semaines après une installation indépendante.
\end{enumerate}
