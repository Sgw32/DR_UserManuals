\chapter{Principe de fonctionnement} \label{ch:operating-principle}

Les tableaux \ReplicaGenOne{} réutilisent le boîtier Volkswagen d'origine, les connecteurs CE~1 ou CE~2 du constructeur et, selon la configuration, soit le câble de compteur de vitesse mécanique soit un capteur de vitesse électronique.
Les cartes principales \ReplicaGenOneShort{} reposent sur un circuit imprimé en fibre de verre peuplé de composants discrets pilotés par un microcontrôleur ATmega~2560 et des pilotes d'afficheurs MAX~7219.

\ReplicaNextLong{} s'appuie sur un système sur puce ESP32-S3 et introduit un boîtier nouvellement fabriqué en impression SLA, une façade et un capot repensés ainsi qu'une carte adaptatrice de connecteur.
L'afficheur \ReplicaNextShort{} est rétroéclairé par des LED adressables WS2812 montées derrière la façade, et le faisceau associé inclut d'office le capteur de vitesse électronique.

Les deux générations partagent la même disposition d'affichage et les mêmes pages MFA, ce qui garantit des procédures d'installation et une utilisation quotidienne familières d'une révision matérielle à l'autre.
