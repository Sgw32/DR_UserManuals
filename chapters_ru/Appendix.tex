\chapter{Справочные таблицы}\label{appendix:reference-ru}

\section{Команды классической \ReplicaGenOne{}}

Прошивка классической Replica использует те же команды, что и \ReplicaNextShort{}.
Команды 31--33 (управление цветом) активны только на \ReplicaNextShort{}; остальные команды работают в обеих поколениях.

\begin{table}[htbp]
    \centering
    \caption{Основные команды конфигурации для классических приборок \ReplicaGenOne{}.}
    \label{tbl:replica-commands-ru}
    {\scriptsize
    \begin{tblr}{
        colspec = {Q[c,0.12\linewidth,cmd=\seqsplit,font=\ttfamily] Q[l,0.32\linewidth,cmd=\seqsplit] Q[l,0.52\linewidth]},
        rowsep = 2pt,
    }
        \toprule
        \SetCell{cmd=\relax}\textbf{Команда} & \SetCell{cmd=\relax}\textbf{Название} & \textbf{Описание} \\
        \midrule
        22 (или 0) & PARAMETER\_RPMCOEFFICIENT & Калибровочный коэффициент оборотов. \\
        1  & PARAMETER\_SPEEDCOEFFICIENT & Калибровка скорости. \\
        2  & PARAMETER\_COOLANTTHERMISTORB & Бета-коэффициент датчика ОЖ. \\
        3  & PARAMETER\_OILTHERMISTORB & Бета-коэффициент датчика масла. \\
        4  & PARAMETER\_AIRTHERMISTORB & Бета-коэффициент датчика наружного воздуха. \\
        5  & PARAMETER\_TANKMINRESISTANCE & Минимальное сопротивление датчика топлива. \\
        6  & PARAMETER\_TANKMAXRESISTANCE & Максимальное сопротивление датчика топлива. \\
        7--10 & PARAMETER\_TAU\_\textit{X} & Константы фильтрации для ОЖ, масла, воздуха и уровня топлива. \\
        11 & PARAMETER\_MILEAGE & Общий пробег. \\
        12 & PARAMETER\_DAILY\_MILEAGE & Суточный пробег. \\
        13 & PARAMETER\_AUTO\_BRIGHTNESS & Включение автоматической яркости. \\
        14 & PARAMETER\_BRIGHTNESS\_LEVEL & Ручной уровень яркости. \\
        15 & PARAMETER\_TANK\_CAPACITY & Ёмкость бака. \\
        16 & PARAMETER\_MFA\_STATE & Активная страница MFA. \\
        17 & PARAMETER\_BUZZER\_OFF & Отключение зуммера (только Replica). \\
        18 & PARAMETER\_MAX\_RPM & Диапазон тахометра (по умолчанию 7000). \\
        19--21 & PARAMETER\_NORMAL\_RESISTANCE\_\textit{X} & Сопротивления датчиков при \SI{25}{\celsius} для ОЖ, масла и наружного воздуха. \\
        23 & PARAMETER\_DOT\_OFF & Поведение разделителя часов. \\
        24 & PARAMETER\_BACKLIGHT\_ON & Включение подсветки при ближнем свете. \\
        25 & PARAMETER\_M\_D\_FILTER & Константа медианного фильтра. \\
        26 & PARAMETER\_COOLANT\_MAX\_R & Порог датчика ОЖ для полной шкалы. \\
        27 & PARAMETER\_COOLANT\_MIN\_R & Порог датчика ОЖ для отметки «1~bar». \\
        31--33 & PARAMETER\_MAINCOLOR\_[RGB] & Компоненты цвета интерфейса (только \ReplicaNextShort{}). \\
        37 & PARAMETER\_RPM\_FILTER & Агрессивность фильтра оборотов. \\
        128 & PARAMETER\_READ\_ADDITION & Добавьте 128 для чтения значения команды. \\
        255 & PARAMETER\_SET\_HOUR & Установка часов. \\
        254 & PARAMETER\_SET\_MINUTE & Установка минут. \\
        253 & PARAMETER\_RESET\_DAILY\_MILEAGE & Сброс суточного пробега. \\
        252 & PARAMETER\_RESET\_DIGITAL & Заводской сброс параметров. \\
        \bottomrule
    \end{tblr}}
\end{table}

\section{Значения по умолчанию для \ReplicaGenOneShort{}}

\begin{table}[htbp]
    \centering
    \caption{Настройки по умолчанию для классической \ReplicaGenOne{}.}
    \label{tbl:replica-defaults-ru}
    {\scriptsize
    \begin{tblr}{
        colspec = {Q[c,0.22\linewidth,cmd=\seqsplit,font=\ttfamily] Q[c,0.16\linewidth,cmd=\seqsplit] Q[l,0.46\linewidth]},
        rowsep = 2pt,
    }
        \toprule
        \SetCell{cmd=\relax}\textbf{Параметр} & \SetCell{cmd=\relax}\textbf{Значение} & \textbf{Примечание} \\
        \midrule
        PARAMETER\_RPMCOEFFICIENT & 3000 &  \\
        PARAMETER\_SPEEDCOEFFICIENT & 100 &  \\
        PARAMETER\_COOLANTTHERMISTORB & 4000 &  \\
        PARAMETER\_OILTHERMISTORB & 4000 &  \\
        PARAMETER\_AIRTHERMISTORB & 3812 & Для поколения~2 — 3600. \\
        PARAMETER\_TANKMINRESISTANCE & 35 & \ohm. \\
        PARAMETER\_TANKMAXRESISTANCE & 265 & \ohm. \\
        PARAMETER\_TAU\_COOLANT & 2 &  \\
        PARAMETER\_TAU\_OIL & 2 &  \\
        PARAMETER\_TAU\_AIR & 2 &  \\
        PARAMETER\_TAU\_TANK & 2 &  \\
        PARAMETER\_MILEAGE & Зависит от автомобиля & Сохраняет текущий пробег. \\
        PARAMETER\_DAILY\_MILEAGE & 0 &  \\
        PARAMETER\_AUTO\_BRIGHTNESS & 1 & Включена. \\
        PARAMETER\_BRIGHTNESS\_LEVEL & 7 или 13 & Типичные значения для поколений~1/1.5. \\
        PARAMETER\_TANK\_CAPACITY & 63 & Литры. \\
        PARAMETER\_MFA\_STATE & 0 &  \\
        PARAMETER\_BUZZER\_OFF & 1 & Зуммер отключён. \\
        PARAMETER\_MAX\_RPM & 8000 & В ранних версиях — 7000. \\
        PARAMETER\_NORMAL\_RESISTANCE\_COOLANT & 1000 & \si{\ohm} при \SI{25}{\celsius}. \\
        PARAMETER\_NORMAL\_RESISTANCE\_OIL & 1000 & \si{\ohm} при \SI{25}{\celsius}. \\
        PARAMETER\_NORMAL\_RESISTANCE\_AMB & 2991 & \si{\ohm} при \SI{25}{\celsius}. \\
        PARAMETER\_DOT\_OFF & 0 & Мигание разделителя. \\
        PARAMETER\_BACKLIGHT\_ON & 1 & Включена. \\
        PARAMETER\_M\_D\_FILTER & 65535 &  \\
        PARAMETER\_COOLANT\_MAX\_R & 120 & \si{\celsius}. \\
        PARAMETER\_COOLANT\_MIN\_R & 60 & \si{\celsius}. \\
        PARAMETER\_MAINCOLOR\_[RGB] & -- & Команды цвета не активны на классической Replica. \\
        PARAMETER\_RPM\_FILTER & 70 &  \\
        PARAMETER\_UPTIME & 0 &  \\
        \bottomrule
    \end{tblr}}
\end{table}

\section{Журнал изменений}\label{app:change-log-ru}

\begin{table}[htbp]
    \centering
    \caption{Лист регистрации изменений документа.}
    \label{tbl:change-log-ru}
    {\scriptsize
    \begin{tblr}{
        colspec = {Q[c,0.12\linewidth] Q[l] Q[c,0.2\linewidth]},
        rowsep = 2pt,
    }
        \toprule
        \textbf{Изменение} & \textbf{Затронутые листы} & \textbf{Дата} \\
        \midrule
        1 & 04.10.2022 & 04~октября~2022 \\
        2 & 31.08.2023 & 31~августа~2023 \\
        3 & 05.08.2024 & 05~августа~2024 \\
        4 & Внедрён LaTeX-документ. & 22.09.2025 \\
        \bottomrule
    \end{tblr}}
\end{table}
