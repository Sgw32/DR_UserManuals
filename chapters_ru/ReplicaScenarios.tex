\chapter{Типовые ситуации при настройке \ReplicaGenOne{}}\label{ch:replica-scenarios-ru}

Перед диагностикой убедитесь, что у вас классическая приборка \ReplicaGenOne{} (см. \autoref{ch:replica-setup-ru}).
Панели \ReplicaNextLong{} используют Wi-Fi-портал и описаны в \autoref{ch:replica-next-scenarios-ru}.

\begin{description}
    \item[Bluetooth-модуль не обнаружен] Подключайтесь к интерфейсу Bluetooth Classic приборки (обычно виден как \texttt{Digifiz}).
    Приложение Serial Bluetooth Terminal для Android остаётся рекомендуемым инструментом: установите символ конца строки LF и избегайте BLE-сканеров, которые не видят модуль.
    \item[iPhone или iPad не подключается] Приборки \ReplicaGenOneShort{} используют Bluetooth~2.0 и несовместимы с iOS. Используйте Android-смартфон или компьютер с терминалом последовательного Bluetooth.
    \item[Команды игнорируются в прошивке 2024+] Разблокируйте парсер команд, отправив \verb|234 123|, затем повторите нужную последовательность. Сохраните часто используемые команды в быстрых кнопках Serial Bluetooth Terminal.
    \item[Неверная скорость] Подключитесь через Serial Bluetooth Terminal, разгонитесь до показаний \SI{100}{\kilo\metre\per\hour} и зафиксируйте скорость по GPS. Отправьте \verb|1 <значение GPS>| (например, \verb|1 85|), чтобы задать \paramname{PARAMETER\_SPEEDCOEFFICIENT} по проверенному значению.
    \item[Неверные обороты] В прошивках до 2024 года используется \verb|0 <значение>|, в текущих — \verb|22 <значение>|. Для двигателей Audi обычно подходит \verb|22 3000|; при необходимости делите или умножайте значение (\verb|22 1500|, \verb|22 6000|), пока показания не совпадут с тахометром.
    \item[Нужно увеличить яркость] Отключите автоматический режим командой \verb|13 0| и увеличьте ручной уровень через \verb|14 <значение>|. Значения 45--55 заметно повышают яркость; избегайте уровней выше 60, чтобы сохранить ресурс светодиодов. Верните фотодатчик в работу командой \verb|13 1|.
    \item[Настройка часов] Отправьте через Serial Bluetooth Terminal команды \verb|255 <часы>| и \verb|254 <минуты>|. Примеры: \verb|255 23| и \verb|254 55| задают 23:55; \verb|255 14|, \verb|254 30| — 14:30; \verb|255 2|, \verb|254 28| — 02:28.
    \item[Проблемы с уровнем топлива] Перед проверками снимите клеммы аккумулятора.
    \begin{itemize}
        \item Если показания «плавают» от 60 до 0, измерьте сопротивление датчика между выводом жгута и массой; рабочий диапазон обычно \SIrange{30}{300}{\ohm}. Очистите разъём и убедитесь, что сигнал поступает на плату.
        \item Если шкала постоянно «полная», найдите короткое замыкание на массу ниже \SI{5}{\ohm} и устраните его.
        \item Если показания не меняются, сравните сопротивление датчика при полном и пустом баке. При неизменном значении замените датчик.
    \end{itemize}
    \item[Расход топлива кажется неверным] Канал расхода эмулируется, если не установлен датчик давления во впускном коллекторе. Рассматривайте показания как ориентировочные.
    \item[Индикатор ОЖ неточен] Отрегулируйте \paramname{PARAMETER\_COOLANT\_MIN\_R} и \paramname{PARAMETER\_COOLANT\_MAX\_R}. Например, \verb|27 30| сдвигает отметку «1~bar» к \SI{30}{\celsius}.
    \item[Нет температуры масла или воздуха] Значение \texttt{-999} или «зависшее» показание указывает на проблему с датчиком. При отключённом аккумуляторе и холодном моторе измерьте сопротивление между выводом датчика и массой. Датчик масла должен показывать около \SI{2}{\kilo\ohm} \ensuremath{\pm}\SI{0.3}{\kilo\ohm}, датчик воздуха — около \SI{10}{\kilo\ohm} \ensuremath{\pm}\SI{2}{\kilo\ohm}. Подкорректируйте \paramname{PARAMETER\_NORMAL\_RESISTANCE\_OIL} (команда~20) или \paramname{PARAMETER\_NORMAL\_RESISTANCE\_AMB} (команда~21), если требуется точная настройка. При сохранении проблемы снимите лог \verb|adc 0| и передайте его в службу поддержки PHOL-LABS Kft.
\end{description}

Если неисправность не устраняется, соберите сырые данные датчиков командой \verb|adc 0| и передайте их разработчикам приборной панели для анализа.
