\chapter{Технические характеристики}\label{ch:technical-specs-ru}

Приборная панель \ReplicaGenOne{} не потребляет ток в режиме ожидания при выключенном питании.
Модель \ReplicaNextShort{} потребляет около 13~мА от шины +12~В при выключенном зажигании, что следует учитывать при длительном хранении автомобиля.
Обе поколения надёжно работают от бортовой сети в диапазоне 9--16~В постоянного тока.

\section{Измерительные возможности}
\begin{itemize}
    \item \textbf{Скорость автомобиля:} измеряется штатным тросовым либо электронным датчиком скорости.
    Систематическая погрешность составляет 10~км/ч, относительная — 3~км/ч, индикация насыщается на 999~км/ч (или mph для имперских единиц).
    \item \textbf{Обороты двигателя:} вычисляются по сигналу зажигания через каскад с оптопарой, RC-цепью 430~нФ/1{,}2~k\ensuremath{\Omega} и диодным ограничителем.
    Абсолютная и относительная погрешности не превышают 200~об/мин.
    \item \textbf{Уровень топлива:} считывается с резистивного датчика бака с неопределённостью около 10~литров.
    \item \textbf{Температура охлаждающей жидкости:} отображается качественно по стандартному термистору, подключённому через штатный жгут; числовые значения не выводятся.
    \item \textbf{Часы:} точность хода — до одной минуты.
    \item \textbf{Световые индикаторы:} указатели поворота, дальний свет, предупреждения по давлению масла, статус генератора, стояночный тормоз, подогрев заднего стекла или свечи накала дизеля, а также передние и задние противотуманные фары.
\end{itemize}
