\chapter{Условия эксплуатации и меры безопасности}\label{ch:safety-ru}

\section{Допустимые условия среды}
\begin{itemize}
    \item Приборная панель работает в диапазоне температур от \(-40\,^{\circ}\mathrm{C}\) до \(+70\,^{\circ}\mathrm{C}\) при относительной влажности до 95~\%.
    \item Щиток может оставаться установленным в автомобиле круглый год, включая длительные периоды стоянки.
\end{itemize}

\section{Меры безопасности}
\begin{enumerate}
    \item Приборка Digifiz — устройство формата DIY, собираемое и интегрируемое энтузиастами. Соблюдайте общие правила электробезопасности при работе с ней.
    \item Изделие предназначено для личных проектов владельцев автомобилей.
    \item Показания не имеют метрологической сертификации, хотя соответствуют заявленным спецификациям на момент выпуска.
    \item Используйте приборку только принимая ответственность за установку и безопасность дорожного движения.
    \item Если показания вызывают сомнения, сверяйте их со штатными приборами автомобиля или внешними измерительными устройствами.
    \item Не подключайте выходы приборной панели к системам автоматического управления автомобилем.
    \item Авторы не несут ответственности за последствия установки или использования приборки, включая штрафы и ДТП. Неисправности, заявленные в гарантийный период (один год при установке совместно с авторами и две недели при самостоятельной установке), будут устранены.
    \item Перечисленные в \Cref{ch:technical-specs-ru} функциональные возможности гарантируются в течение года при установке под контролем авторов и в течение двух недель после самостоятельной установки.
\end{enumerate}
