\chapter{Типовые ситуации при настройке \ReplicaNextShort{}}\label{ch:replica-next-scenarios-ru}

\begin{description}
    \item[Точка доступа не видна] Подойдите ближе к автомобилю и убедитесь, что он стоит на открытой площадке. Отключите мобильные данные, удалите устаревшие Wi-Fi-профили и заново подключитесь к \texttt{Digifiz\_AP} (или \texttt{PHOL-LABS2}).
    \item[Ошибка 404 по адресу \texttt{192.168.4.1}] Выключите мобильный интернет на телефоне или ноутбуке и обновите страницу. Механизм captive-порталов в Android/iOS часто мешает, пока не отключён сотовый модем.
    \item[Обновление прошивки] Откройте вкладку \emph{WiFi} и выберите файл \texttt{Digifiz.bin}. Актуальные релизы опубликованы по ссылке ниже.
        \displayurl{https://github.com/Sgw32/DigifizReplica/releases}
        Нажмите \emph{Upload}. Первая попытка может завершиться с ошибкой — при необходимости повторите. Удачная прошивка перенаправляет на страницу подтверждения. Перед обновлением запишите пробег и восстановите его командой \verb|11 <пробег>|.
    \item[Команды игнорируются] Обновите страницу браузера, вернитесь на вкладку \emph{Control} и отправьте команду повторно. Убедитесь, что после ввода нажата кнопка \emph{Process}.
    \item[Неверная скорость] Подключитесь по Wi-Fi, разгонитесь до показаний \SI{100}{\kilo\metre\per\hour}, сравните их со скоростью по GPS и выполните команду \verb|1 <значение GPS>| (например, \verb|1 85|), чтобы задать \paramname{PARAMETER\_SPEEDCOEFFICIENT} по проверенному значению.
    \item[Неверные обороты] Отрегулируйте \paramname{PARAMETER\_RPMCOEFFICIENT}. В старых прошивках используется \verb|0 <значение>|, в текущих — \verb|22 <значение>|. Пример: \verb|22 1500| уменьшает показания вдвое относительно \verb|22 3000|.
    \item[Слишком тусклый дисплей] Отключите автоматическую яркость командой \verb|13 0|, затем увеличьте ручной уровень (например, \verb|14 50|). Экспериментируйте со значениями 45--55, избегайте уровней выше 60, чтобы продлить срок службы светодиодов.
    \item[Настройка часов] Используйте веб-терминал (или Serial Bluetooth Terminal в старых версиях) и отправьте \verb|255 <часы>|, затем \verb|254 <минуты>|. Пример: \verb|255 23| и \verb|254 55| установят 23:55.
    \item[Завис показания уровня топлива] Отключите аккумулятор и измерьте сопротивление между выводом датчика топлива и массой кузова. Рабочие значения обычно находятся в диапазоне \SIrange{30}{300}{\ohm}. Устраните замыкания ниже \SI{5}{\ohm} и обрывы. Если сопротивление меняется корректно, но шкала не реагирует, снимите показания \verb|adc 0| при нескольких уровнях топлива и отправьте их в PHOL-LABS Kft.
    \item[Неточные данные по расходу топлива] Опциональный датчик расхода генерирует эмулированные данные и без датчика давления во впуске работает нестабильно. Рассматривайте показания как экспериментальные.
    \item[Температура охлаждающей жидкости вне диапазона] Настройте \paramname{PARAMETER\_COOLANT\_MIN\_R} и \paramname{PARAMETER\_COOLANT\_MAX\_R}. Пример: \verb|27 30| снижает порог «1~bar» до \SI{30}{\celsius}.
    \item[Не отображается температура масла или воздуха] При отключённом аккумуляторе и холодном двигателе измерьте сопротивление датчиков. Датчик масла должен показывать около \SI{2}{\kilo\ohm} \ensuremath{\pm}\SI{0.3}{\kilo\ohm}, датчик наружного воздуха — около \SI{10}{\kilo\ohm} \ensuremath{\pm}\SI{2}{\kilo\ohm}. Подкорректируйте \paramname{PARAMETER\_NORMAL\_RESISTANCE\_OIL} (команда~20) или \paramname{PARAMETER\_NORMAL\_RESISTANCE\_AMB} (команда~21); меньшие значения уменьшают отображаемую температуру, большие — увеличивают. При сохранении проблемы снимите вывод \verb|adc 0| и обратитесь в PHOL-LABS Kft.
    \item[Смена цвета интерфейса] Используйте команды 31--33 для установки значений RGB. В новых версиях прошивки доступны визуальные элементы настройки, поэтому обновляйте приборку регулярно.
\end{description}
