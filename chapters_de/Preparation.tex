\chapter{Arbeitsvorbereitung und Arbeitsablauf}\label{ch:preparation}

\section{Vorbereitung des Fahrzeugs}
Gehen Sie beim Austausch des Serienkombiinstruments gegen ein Digifiz-Armaturenbrett in folgender Reihenfolge vor:
\begin{enumerate}
    \item Entfernen Sie die Kunststoffverkleidung über den Pedalen und im unteren Armaturenbereich, um das originale Kombiinstrument freizulegen.
    \item Klemmen Sie die Fahrzeugbatterie ab.
    \item Ziehen Sie den Kabelbaum vom serienmäßigen Kombiinstrument ab.
    \item Lösen Sie, falls vorhanden, die mechanische Tachowelle.
    \item Schrauben Sie das Kombiinstrument aus seinen Haltern und entnehmen Sie es vorsichtig aus dem Fahrzeug.
    \item Verlegen Sie die mitgelieferten Temperatur- und Geschwindigkeitssensor-Kabelbäume wie erforderlich.
    \item Setzen Sie das Digifiz-Kombiinstrument in die Führungsschienen der Halterung ein und sichern Sie es mit Schrauben.
    \item Installieren Sie bei \ReplicaNextLong{} die Volkswagen-MFA-Sensoren (oder gleichwertige Sensoren) und führen Sie deren Leitungen zu den CE~1-/CE~2-Steckverbindern.
    \item Verbinden Sie bei den Modellen \texttt{GACS}/\texttt{GARS}/\texttt{DARS}/\texttt{DACS} die markierten Leitungen \texttt{MFA\_MODE}, \texttt{MFA\_RESET}, \texttt{MFA\_BLOCK} und Handbremse manuell, sofern diese Kontakte im Fahrzeugkabelbaum fehlen. Bei der zweiten Generation \ReplicaNextShort{} sind diese Signale standardmäßig intern geführt.
    \item Stecken Sie die Kabelbäume in das Kombiinstrument.
    \item Montieren Sie den elektronischen Geschwindigkeitssensor oder schließen Sie die mechanische Welle wieder an.
    \item Bauen Sie die Armaturenverkleidung und die Pedalabdeckung in umgekehrter Reihenfolge wieder ein.
\end{enumerate}

\section{Bedienung des Kombiinstruments}
\begin{itemize}
    \item Das Kombiinstrument schaltet sich mit der Zündung automatisch ein. Die Standlichtschalter steuern die Hintergrundbeleuchtung.
    \item Beim Start leuchtet die komplette Geschwindigkeitsanzeige auf, während die internen Diagnosen das Drehzahlmodell stabilisieren; anschließend stellt sich das Display auf die aktuelle Leerlaufdrehzahl ein.
    \item Sobald sich das Fahrzeug bewegt, zeigt das System die in \Cref{ch:technical-specs} aufgeführten Parameter an.
\end{itemize}

\subsection{MFA-Funktionen}
Es stehen sechs MFA-Seiten zur Verfügung:
\begin{enumerate}
    \item Tägliche Betriebszeit.
    \item Fahrstrecke.
    \item Kraftstoffverbrauch (in der ersten \Replica{}-Revision nicht implementiert).
    \item Durchschnittsgeschwindigkeit (angezeigt als Wert multipliziert mit zehn).
    \item Motoröltemperatur (externer Kabelbaum erforderlich).
    \item Außentemperatur (externer Kabelbaum erforderlich).
\end{enumerate}
Bei \ReplicaGenOneShort{}-Kombiinstrumenten blättert eine kapazitive Berührfläche hinter dem VW-Emblem durch die Seiten; \ReplicaNextShort{} nutzt hierfür einen externen Lenkstockschalter. Berührungsdauern wirken sich wie folgt aus:
\begin{itemize}
    \item Kurzer Tastendruck (\(<1\)~s): Weiter zur nächsten MFA-Funktion.
    \item Mittlerer Tastendruck (1--3~s, sofern kein Lenkstockschalter montiert ist): Umschalten zwischen den MFA-Speicherblöcken; der Wechsel wird im Display angezeigt.
    \item Langer Tastendruck (3--7~s): Rücksetzen der aktiven MFA-Funktion (betrifft Verbrauch, Strecke, Laufzeit und Durchschnittsgeschwindigkeit).
\end{itemize}

\subsection{Hintergrundbeleuchtung und Anzeigefelder}
Das \ReplicaGenOneShort{}-Kombiinstrument bietet oberhalb des Standlichtschalters eine manuelle Helligkeitseinstellung; \ReplicaNextShort{} verlässt sich auf eine automatische Regelung durch eine Fotodiode. Manuelle Übersteuerungen lassen sich über die in \Cref{ch:replica-setup,ch:replica-next-setup} beschriebenen Wartungsschnittstellen konfigurieren.

Das Layout des horizontalen Anzeigeblocks sowie die Bildschirmlegende zeigt \autoref{fig:indicator-layout}.

\begin{figure}[htbp]
    \centering
    \begin{subfigure}{0.48\textwidth}
        \includegraphics[width=\linewidth]{digifiz_manual/image017.png}
        \caption{Anzeigelayout während des Selbsttests beim Einschalten.}
    \end{subfigure}\hfill
    \begin{subfigure}{0.48\textwidth}
        \includegraphics[width=\linewidth]{digifiz_manual/image018.png}
        \caption{Legende für die horizontale Anzeigenreihe.}
    \end{subfigure}
    \caption{Anzeigeschema des Kombiinstruments.}
    \label{fig:indicator-layout}
\end{figure}

\subsection{Konfigurationsschnittstellen}
\begin{itemize}
    \item Klassische \ReplicaGenOne{}-Einheiten verfügen über ein Bluetooth-2.0-Modul (bzw. BLE-kompatibel). Installieren Sie die App \emph{Serial Bluetooth Terminal} aus dem Google Play Store, koppeln Sie das Kombiinstrument und geben Sie Befehle direkt aus der Terminalansicht ein. Apple iOS-Geräte können keine Verbindung zu diesem Modul herstellen.
    \item \ReplicaNextShort{} stellt einen integrierten Wi-Fi-Zugangspunkt und ein Konfigurationsportal bereit, das in \Cref{ch:replica-next-setup} beschrieben ist. Deaktivieren Sie während der Verbindung die mobilen Daten, damit das Captive Portal zuverlässig geladen wird.
\end{itemize}
Beide Generationen können außerdem auf der Werkbank über die USBasp-Programmierschnittstelle mit Strom versorgt und konfiguriert werden.
