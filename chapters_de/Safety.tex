\chapter{Betriebsbedingungen und Sicherheitshinweise}\label{ch:safety}

\section{Umweltbedingungen}
\begin{itemize}
    \item Das Kombiinstrument arbeitet in einem Temperaturbereich von \(-40\,^{\circ}\mathrm{C}\) bis \(+70\,^{\circ}\mathrm{C}\) bei einer relativen Luftfeuchte von bis zu 95~\%.
    \item Das Armaturenbrett darf das ganze Jahr über im Fahrzeug verbaut bleiben, auch wenn der Wagen länger abgestellt wird.
\end{itemize}

\section{Sicherheits- und Haftungshinweise}
\begin{enumerate}
    \item Das Digifiz-Kombiinstrument ist ein Do-it-yourself-Gerät, das von Enthusiasten zusammengebaut und integriert wird. Beachten Sie bei der Arbeit daran die allgemeinen Regeln der Elektrosicherheit.
    \item Das Produkt ist für die privaten Projekte der Fahrzeughalter bestimmt.
    \item Die Anzeigen sind nicht geeicht oder messtechnisch verifiziert, entsprechen aber zum Zeitpunkt der Veröffentlichung den angegebenen Spezifikationen.
    \item Nutzen Sie das Kombiinstrument nur, wenn Sie Verantwortung für den Einbau und die Verkehrssicherheit übernehmen.
    \item Sollten die angezeigten Werte zweifelhaft sein, gleichen Sie sie mit den serienmäßigen Instrumenten oder externen Messgeräten ab.
    \item Verwenden Sie die Signalausgänge des Kombiinstruments nicht für automatische Fahrzeugsysteme.
    \item Die Autoren übernehmen keine Haftung für Folgen aus Einbau oder Nutzung des Kombiinstruments, einschließlich Verkehrsverstößen oder Unfällen. Fehler, die innerhalb der Garantiezeit gemeldet werden (ein Jahr bei gemeinsam mit den Autoren durchgeführten Installationen und zwei Wochen bei eigenständigen Installationen), werden behoben.
    \item Die in \Cref{ch:technical-specs} aufgeführten Funktionen werden für ein Jahr bei betreuter Installation und für zwei Wochen nach eigenständigem Einbau gewährleistet.
\end{enumerate}
