\chapter{Einrichtung und Wartung der \ReplicaGenOne{}}\label{ch:replica-setup}

Dieses Kapitel bezieht sich auf das klassische \ReplicaGenOne{}-Kombiinstrument in \autoref{fig:replica-classic}. Entspricht Ihr Armaturenbrett dem \ReplicaNextLong{}-Layout, lesen Sie das vorherige Kapitel.

\begin{figure}[htbp]
    \centering
    \begin{subfigure}{0.46\textwidth}
        \includegraphics[width=\linewidth]{digifiz_manual/image046.png}
        \caption{Klassische \ReplicaGenOne{} mit kantigem Rahmen.}
    \end{subfigure}\hfill
    \begin{subfigure}{0.46\textwidth}
        \includegraphics[width=\linewidth]{digifiz_manual/image047.png}
        \caption{Späterer Bausatz mit abgerundeter Front.}
    \end{subfigure}
    \caption{Erscheinungsbild des \ReplicaGenOne{}-Kombiinstruments.}
    \label{fig:replica-classic}
\end{figure}

\section{Handhabung und Pflege der Anzeige}
\begin{itemize}
    \item Die Frontscheibe aus Plexiglas mit UV-Druck verkratzt leicht. Vermeiden Sie Kontakt mit scharfen oder scheuernden Gegenständen.
    \item Oberflächenschäden sind kosmetisch und nicht von der Garantie abgedeckt. Fordern Sie Ersatzteile bei PHOL-LABS Kft an, falls das Druckbild der Anzeige beschädigt ist.
\end{itemize}

\section{Batterie der Echtzeituhr}
Im Kombiinstrument arbeitet eine DS3231-Echtzeituhr mit einer CR2032-Knopfzelle. Die Batterie hält typischerweise etwa vier Jahre. Ist sie entladen, setzt sich die Uhr bei jedem Einschalten zurück. Entfernen Sie die vordere und/oder hintere Abdeckung, lassen Sie die Kabelbäume angeschlossen und tauschen Sie die Knopfzelle aus. Entsorgen Sie die verbrauchte Batterie gemäß den lokalen Vorschriften.

\section{Firmwarepflege mit USBasp}
Jeder Bausatz enthält einen bereits im Gehäuse angeschlossenen USBasp-Programmieradapter (\autoref{fig:usbasp-cable}). Installieren Sie vor dem Flashen einen passenden USBasp-Treiber, z.~B. von folgender Adresse:
\displayurl{https://myrobot.ru/downloads/driver-usbasp-v-2.0-usb-isp-windows-7-8-10-xp.php}
Der Programmer versorgt das Kombiinstrument mit Strom, sobald er mit dem Computer verbunden ist, sodass Prüfungen auf der Werkbank möglich sind.

\begin{figure}[htbp]
    \centering
    \includegraphics[width=0.32\textwidth]{digifiz_manual/image048.png}
    \caption{Lage des USBasp-Kabels innerhalb der \ReplicaGenOne{}.}
    \label{fig:usbasp-cable}
\end{figure}

Die Firmware wird mit \texttt{avrdude} per folgendem Befehl übertragen (ersetzen Sie den Dateinamen bei Bedarf):

\begin{verbatim}
avrdude -c usbasp -p m2560 -e \
    -U lfuse:w:0xff:m -U hfuse:w:0x99:m -U efuse:w:0xff:m \
    -U flash:w:Digifiz.ino.mega.hex
\end{verbatim}

Drücken Sie nach erfolgreichem Flashen die vordere Touchfläche vier- bis fünfmal, um die Speicherblöcke zu initialisieren. Bleiben sie leer, wiederholen Sie den Flashvorgang oder senden Sie per Bluetooth den Befehl \verb|252 0| für einen Werksreset. Einsatzbereite Firmware-Images sind unter folgendem Link verfügbar:
\displayurl{https://github.com/Sgw32/DigifizReplica}

\section{Bluetooth-Konfiguration}
Die meisten Parameter werden per Bluetooth über ein Android-Smartphone und die App Serial Bluetooth Terminal eingestellt. Laden Sie die Anwendung vor dem Koppeln über folgenden Link herunter:
\displayurl{https://play.google.com/store/apps/details?id=de.kai_morich.serial_bluetooth_terminal&hl=en&gl=US}
Geräte mit iOS können sich nicht mit dem klassischen Bluetooth-2.0-Modul verbinden.

\begin{itemize}
    \item Koppeln Sie sich mit der Bluetooth-Classic-Schnittstelle des Kombiinstruments und nicht mit reinen BLE-Geräten.
    \item Stellen Sie in Serial Bluetooth Terminal das Zeilenende auf LF und deaktivieren Sie CR+LF, bevor Sie Befehle senden.
\end{itemize}

\begin{figure}[htbp]
    \centering
    \includegraphics[width=0.32\textwidth]{digifiz_manual/image049.png}
    \caption{Empfohlene Einstellungen in Serial Bluetooth Terminal.}
    \label{fig:sbt-settings}
\end{figure}

Geben Sie Befehle als Leerzeichen-getrennte Paare \verb|<Nummer> <Wert>| ein. Um beispielsweise einen Kilometerstand von 123\,456~km zu speichern, senden Sie \verb|11 123456|. Addieren Sie 128 zur Befehlsnummer, um den aktuellen Wert abzurufen (\verb|129 0| gibt den Geschwindigkeitsfaktor zurück). Der Diagnosebefehl \verb|adc 0| liefert Rohwerte der Sensoren und unterstützt die Entwickler bei der Fehlersuche.

\section{Konfigurationsparameter}
Die wichtigsten Bluetooth-Befehle sind in \autoref{tbl:replica-classic-commands} aufgeführt. Standardwerte für Kombiinstrumente der Generationen~1/1.5 und~2 fasst \autoref{tbl:replica-defaults} zusammen. Die Befehle~31--33 wirken ausschließlich auf \ReplicaNextShort{}-Geräte; bei der klassischen \ReplicaGenOneShort{} haben sie keine Funktion.

{\scriptsize
\begin{longtblr}[
    caption = {Konfigurationsbefehle der klassischen \ReplicaGenOne{}.},
    label = {tbl:replica-classic-commands},
]{
    colspec = {Q[c,0.14\linewidth] Q[l,0.36\linewidth] Q[l,0.5\linewidth]},
    rowsep = 2pt,
}
    \toprule
    \textbf{ID} & \textbf{Name} & \textbf{Beschreibung} \\
    \midrule
    22 (oder 0) & \paramname{PARAMETER\_RPMCOEFFICIENT} & Kalibrierfaktor der Motordrehzahl. \\
    1 & \paramname{PARAMETER\_SPEEDCOEFFICIENT} & Kalibrierfaktor für die Geschwindigkeit. \\
    2 & \paramname{PARAMETER\_COOLANTTHERMISTORB} & Beta-Koeffizient des Kühlmittelthermistors. \\
    3 & \paramname{PARAMETER\_OILTHERMISTORB} & Beta-Koeffizient des Ölthermistors. \\
    4 & \paramname{PARAMETER\_AIRTHERMISTORB} & Beta-Koeffizient des Außentemperaturthermistors. \\
    5 & \paramname{PARAMETER\_TANKMINRESISTANCE} & Minimaler Widerstand des Tankgebers (\si{\ohm}). \\
    6 & \paramname{PARAMETER\_TANKMAXRESISTANCE} & Maximaler Widerstand des Tankgebers (\si{\ohm}). \\
    7 & \paramname{PARAMETER\_TAU\_COOLANT} & Zeitkonstante der Kühlmittelanzeige. \\
    8 & \paramname{PARAMETER\_TAU\_OIL} & Zeitkonstante der Öltemperaturanzeige. \\
    9 & \paramname{PARAMETER\_TAU\_AIR} & Zeitkonstante der Außentemperaturanzeige. \\
    10 & \paramname{PARAMETER\_TAU\_TANK} & Zeitkonstante der Tankanzeige. \\
    11 & \paramname{PARAMETER\_MILEAGE} & Gesamtkilometerstand. \\
    12 & \paramname{PARAMETER\_DAILY\_MILEAGE} & Tageskilometerzähler. \\
    13 & \paramname{PARAMETER\_AUTO\_BRIGHTNESS} & Automatische Helligkeitsregelung aktivieren. \\
    14 & \paramname{PARAMETER\_BRIGHTNESS\_LEVEL} & Manueller Helligkeitswert (0--15). \\
    15 & \paramname{PARAMETER\_TANK\_CAPACITY} & Tankinhalt (Liter). \\
    16 & \paramname{PARAMETER\_MFA\_STATE} & Aktive MFA-Seite. \\
    17 & \paramname{PARAMETER\_BUZZER\_OFF} & Summer deaktivieren (1 = aus, 0 = an). \\
    18 & \paramname{PARAMETER\_MAX\_RPM} & Drehzahlmesserbereich (Standard 8000). \\
    19 & \paramname{PARAMETER\_NORMAL\_RESISTANCE\_COOLANT} & Widerstand des Kühlmittelsensors bei \SI{25}{\celsius}. \\
    20 & \paramname{PARAMETER\_NORMAL\_RESISTANCE\_OIL} & Widerstand des Ölsensors bei \SI{25}{\celsius}. \\
    21 & \paramname{PARAMETER\_NORMAL\_RESISTANCE\_AMB} & Widerstand des Außensensors bei \SI{25}{\celsius}. \\
    23 & \paramname{PARAMETER\_DOT\_OFF} & Verhalten der Uhr-Doppelpunktanzeige (0 blinkt, 1 dauerhaft). \\
    24 & \paramname{PARAMETER\_BACKLIGHT\_ON} & Hintergrundbeleuchtung mit Abblendlicht einschalten. \\
    25 & \paramname{PARAMETER\_M\_D\_FILTER} & Medianfilter-Konstante (Legacy). \\
    26 & \paramname{PARAMETER\_COOLANT\_MAX\_R} & Schwellenwert für die Kühlmittel-Vollskala. \\
    27 & \paramname{PARAMETER\_COOLANT\_MIN\_R} & Schwellenwert für den „1~bar“-Punkt der Kühlmittelanzeige. \\
    31--33 & \paramname{PARAMETER\_MAINCOLOR\_[RGB]} & Farbkanäle der Anzeige (nur \ReplicaNextShort{}). \\
    37 & \paramname{PARAMETER\_RPM\_FILTER} & Stärke der Drehzahlfilterung. \\
    128 & \paramname{PARAMETER\_READ\_ADDITION} & Zur Befehlsnummer addieren, um einen Wert auszulesen. \\
    255 & \paramname{PARAMETER\_SET\_HOUR} & Uhrstunden setzen (24-Stunden-Format). \\
    254 & \paramname{PARAMETER\_SET\_MINUTE} & Uhrminuten setzen. \\
    253 & \paramname{PARAMETER\_RESET\_DAILY\_MILEAGE} & Tageskilometerzähler zurücksetzen. \\
    252 & \paramname{PARAMETER\_RESET\_DIGITAL} & Werksreset und Speicherinitialisierung. \\
    \bottomrule
\end{longtblr}}

Die Schnellwahltasten in Serial Bluetooth Terminal erleichtern Routineaktionen wie das Umschalten der automatischen Helligkeit (\verb|13 0| bzw. \verb|13 1|) oder das Schreiben von Farbwerten. Werte über \SI{60}{\percent} Helligkeit sollten Sie nur kurzzeitig nutzen, um die Lebensdauer der LEDs zu erhalten.
