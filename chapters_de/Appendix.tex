\chapter{Referenztabellen} \label{appendix:reference}

\section{Befehlsübersicht der klassischen \ReplicaGenOne{}}

Die klassische Replica-Firmware teilt sich die meisten Befehle mit der \ReplicaNextShort{}. Die Befehle~31--33 (Farbsteuerung) wirken nur auf \ReplicaNextShort{}-Geräten; alle anderen gelten für beide Generationen.

\begin{table}[htbp]
    \centering
    \caption{Wichtige Konfigurationsbefehle für klassische \ReplicaGenOne{}-Kombiinstrumente.}
    \label{tbl:replica-commands}
    {\scriptsize
    \begin{tblr}{
        colspec = {Q[c,0.12\linewidth,cmd=\seqsplit,font=\ttfamily] Q[l,0.32\linewidth,cmd=\seqsplit] Q[l,0.52\linewidth]},
        rowsep = 2pt,
    }
        \toprule
        \SetCell{cmd=\relax}\textbf{Befehl} & \SetCell{cmd=\relax}\textbf{Name} & \textbf{Beschreibung} \\
        \midrule
        22 (oder 0) & PARAMETER\_RPMCOEFFICIENT & Kalibrierfaktor der Motordrehzahl. \\
        1  & PARAMETER\_SPEEDCOEFFICIENT & Kalibrierfaktor für die Geschwindigkeit. \\
        2  & PARAMETER\_COOLANTTHERMISTORB & Beta-Koeffizient des Kühlmittelthermistors. \\
        3  & PARAMETER\_OILTHERMISTORB & Beta-Koeffizient des Ölthermistors. \\
        4  & PARAMETER\_AIRTHERMISTORB & Beta-Koeffizient des Außentemperaturthermistors. \\
        5  & PARAMETER\_TANKMINRESISTANCE & Minimaler Tankgeberwiderstand. \\
        6  & PARAMETER\_TANKMAXRESISTANCE & Maximaler Tankgeberwiderstand. \\
        7--10 & PARAMETER\_TAU\_\textit{X} & Filterkonstanten für Kühlmittel, Öl, Außentemperatur und Tankinhalt. \\
        11 & PARAMETER\_MILEAGE & Gesamtkilometerstand. \\
        12 & PARAMETER\_DAILY\_MILEAGE & Tageskilometerzähler. \\
        13 & PARAMETER\_AUTO\_BRIGHTNESS & Automatische Helligkeit aktivieren. \\
        14 & PARAMETER\_BRIGHTNESS\_LEVEL & Manuelle Helligkeit. \\
        15 & PARAMETER\_TANK\_CAPACITY & Tankvolumen. \\
        16 & PARAMETER\_MFA\_STATE & Aktiver MFA-Modus. \\
        17 & PARAMETER\_BUZZER\_OFF & Summer deaktivieren (nur Replica). \\
        18 & PARAMETER\_MAX\_RPM & Skalierung des Drehzahlmessers (Standard 7000). \\
        19--21 & PARAMETER\_NORMAL\_RESISTANCE\_\textit{X} & Sensorwiderstände bei \SI{25}{\celsius} für Kühlmittel, Öl und Außentemperatur. \\
        23 & PARAMETER\_DOT\_OFF & Verhalten des Uhr-Doppelpunkts. \\
        24 & PARAMETER\_BACKLIGHT\_ON & Hintergrundbeleuchtung mit Abblendlicht aktivieren. \\
        25 & PARAMETER\_M\_D\_FILTER & Medianfilter-Konstante. \\
        26 & PARAMETER\_COOLANT\_MAX\_R & Schwelle für die Kühlmittel-Vollskala. \\
        27 & PARAMETER\_COOLANT\_MIN\_R & Schwelle für die „1~bar“-Markierung der Kühlmittelanzeige. \\
        31--33 & PARAMETER\_MAINCOLOR\_[RGB] & Farbkanäle der Oberfläche (nur \ReplicaNextShort{}). \\
        37 & PARAMETER\_RPM\_FILTER & Stärke der Drehzahlfilterung. \\
        128 & PARAMETER\_READ\_ADDITION & Zur Befehlsnummer addieren, um einen Wert auszulesen. \\
        255 & PARAMETER\_SET\_HOUR & Uhrstunden setzen. \\
        254 & PARAMETER\_SET\_MINUTE & Uhrminuten setzen. \\
        253 & PARAMETER\_RESET\_DAILY\_MILEAGE & Tageskilometerzähler zurücksetzen. \\
        252 & PARAMETER\_RESET\_DIGITAL & Gespeicherte Parameter auf Werkseinstellungen setzen. \\
        \bottomrule
    \end{tblr}}
\end{table}

\section{Standardwerte der klassischen \ReplicaGenOneShort{}}

\begin{table}[htbp]
    \centering
    \caption{Standardkonfiguration der klassischen \ReplicaGenOne{}.}
    \label{tbl:replica-defaults}
    {\scriptsize
    \begin{tblr}{
        colspec = {Q[c,0.22\linewidth,cmd=\seqsplit,font=\ttfamily] Q[c,0.16\linewidth] Q[l,0.46\linewidth]},
        rowsep = 2pt,
    }
        \toprule
        \SetCell{cmd=\relax}\textbf{Parameter} & \SetCell{cmd=\relax}\textbf{Standard} & \textbf{Hinweise} \\
        \midrule
        PARAMETER\_RPMCOEFFICIENT & 3000 &  \\
        PARAMETER\_SPEEDCOEFFICIENT & 100 &  \\
        PARAMETER\_COOLANTTHERMISTORB & 4000 &  \\
        PARAMETER\_OILTHERMISTORB & 4000 &  \\
        PARAMETER\_AIRTHERMISTORB & 3812 & Generation~2 nutzt 3600. \\
        PARAMETER\_TANKMINRESISTANCE & 35 & \ohm. \\
        PARAMETER\_TANKMAXRESISTANCE & 265 & \ohm. \\
        PARAMETER\_TAU\_COOLANT & 2 &  \\
        PARAMETER\_TAU\_OIL & 2 &  \\
        PARAMETER\_TAU\_AIR & 2 &  \\
        PARAMETER\_TAU\_TANK & 2 &  \\
        PARAMETER\_MILEAGE & Fahrzeugspezifisch & Bisherigen Kilometerstand beibehalten. \\
        PARAMETER\_DAILY\_MILEAGE & 0 &  \\
        PARAMETER\_AUTO\_BRIGHTNESS & 1 & Aktiviert. \\
        PARAMETER\_BRIGHTNESS\_LEVEL & 7 oder 13 & Typische Werte für Gen~1/1.5. \\
        PARAMETER\_TANK\_CAPACITY & 63 & Liter. \\
        PARAMETER\_MFA\_STATE & 0 &  \\
        PARAMETER\_BUZZER\_OFF & 1 & Summer deaktiviert. \\
        PARAMETER\_MAX\_RPM & 8000 & 7000 bei frühen Geräten. \\
        PARAMETER\_NORMAL\_RESISTANCE\_COOLANT & 1000 & \si{\ohm} bei \SI{25}{\celsius}. \\
        PARAMETER\_NORMAL\_RESISTANCE\_OIL & 1000 & \si{\ohm} bei \SI{25}{\celsius}. \\
        PARAMETER\_NORMAL\_RESISTANCE\_AMB & 2991 & \si{\ohm} bei \SI{25}{\celsius}. \\
        PARAMETER\_DOT\_OFF & 0 & Blinkender Doppelpunkt. \\
        PARAMETER\_BACKLIGHT\_ON & 1 & Aktiv. \\
        PARAMETER\_M\_D\_FILTER & 65535 &  \\
        PARAMETER\_COOLANT\_MAX\_R & 120 & \si{\celsius}. \\
        PARAMETER\_COOLANT\_MIN\_R & 60 & \si{\celsius}. \\
        PARAMETER\_MAINCOLOR\_[RGB] & -- & Farbbefehle ohne Wirkung auf der klassischen Replica. \\
        PARAMETER\_RPM\_FILTER & 70 &  \\
        PARAMETER\_UPTIME & 0 &  \\
        \bottomrule
    \end{tblr}}
\end{table}

\section{Änderungshistorie} \label{app:change-log}

\begin{table}[htbp]
    \centering
    \caption{Protokollblatt der Dokumentänderungen.}
    \label{tbl:change-log}
    {\scriptsize
    \begin{tblr}{
        colspec = {Q[c,0.12\linewidth] Q[l] Q[c,0.2\linewidth]},
        rowsep = 2pt,
    }
        \toprule
        \textbf{Änderung} & \textbf{Betroffene Seiten} & \textbf{Datum} \\
        \midrule
        1 & 04.10.2022 & 04.~Okt.~2022 \\
        2 & 31.08.2023 & 31.~Aug.~2023 \\
        3 & 05.08.2024 & 05.~Aug.~2024 \\
        4 & LaTeX-Dokument eingeführt. & 22.~Sep.~2025 \\
        \bottomrule
    \end{tblr}}
\end{table}
