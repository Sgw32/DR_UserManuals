\chapter{Vorsichtsmaßnahmen}\label{ch:precautions}

Dieses Kapitel fasst die verbindlichen Sicherheitsregeln zusammen, die jedem \ReplicaGenOne{}- und \ReplicaNextLong{}-Kombiinstrument beiliegen. Werden sie missachtet, sind beschädigte Elektronik oder unzuverlässige Messwerte die wahrscheinlichste Folge.

\begin{enumerate}
    \item \textbf{Klemmen Sie vor Beginn der Installation die Fahrzeugbatterie ab.} An einem unter Spannung stehenden Kabelbaum zu arbeiten wirkt schneller, doch bereits mehrere Armaturenbretter wurden durch Kurzschlüsse in einem aktiven Kabelstrang zerstört.
    \item \textbf{Legen Sie an die Sensoreingänge niemals eine externe Spannungsquelle an.} Die Kanäle für Kühlmitteltemperatur, Öltemperatur, Außentemperatur und Tankinhalt sind ausschließlich für passive Sensoren ausgelegt. Selbst ein vermeintlich „harmloser“ Test über einen Vorwiderstand brennt die Messtechnik durch.
    \item \textbf{Beachten Sie, dass Generation~1 und~1.5 keine interne Sicherung besitzen.} Das erste Schutzelement ist die 15-A-Sicherung im Volkswagen-Sicherungskasten. Sie reagiert viel zu spät, um das Kombiinstrument vor Verdrahtungsfehlern zu bewahren.
    \item \textbf{Schützen Sie das Gerät vor direkter Sonneneinstrahlung.} Langandauernde Bestrahlung bleicht die LCD-Segmente aus und verringert den Kontrast dauerhaft.
    \item \textbf{Versuchen Sie nicht, die LED-Hintergrundbeleuchtung zu übersteuern.} Die Generationen~1,~1.5 und~2 verwenden eine Konstantstrombeleuchtung. Ist das Bild am Tag zu dunkel, schaffen Sie lieber Abschattung rund um den Instrumententräger, anstatt den Ansteuerstrom zu erhöhen.
    \item \textbf{Achten Sie auf Resonanzen bei mechanisch angetriebenen Tachos.} Mechanische Antriebe neigen bei 40--60~km/h zu Schwingungen. Bauen Sie nach Möglichkeit den mitgelieferten elektronischen Sensor ein\textemdash{}er gehört bei allen aktuellen Gen~1.5- und Gen~2-Sätzen zum Lieferumfang.
    \item \textbf{Planen Sie externe MFA-Bedienelemente für Kombiinstrumente der Generation~2 ein.} Der Touchsensor mit VW-Logo entfiel, daher muss das Umschalten der MFA-Modi über den Lenkstockhebel oder einen anderen externen Schalter erfolgen.
    \item \textbf{Berücksichtigen Sie den Ruhestrom.} Ein Kombiinstrument der Generation~2 entnimmt der Fahrzeugbatterie auch bei ausgeschalteter Zündung etwa 11--13~mA. Dieser Ruheverbrauch lässt sich nicht reduzieren.
    \item \textbf{Die Momentanverbrauchsanzeige ist nicht ab Werk verbaut.} Die Funktion kann gemäß den unten stehenden Anweisungen in Gen~1- und Gen~1.5-Geräten nachgerüstet werden, wurde jedoch noch nicht für Gen~2-Hardware validiert.
        \displayurl{https://www.youtube.com/watch?v=qWqvYc9388U}
\end{enumerate}

\begin{figure}[htbp]
    \centering
    \begin{subfigure}{0.46\textwidth}
        \includegraphics[width=\linewidth]{digifiz_manual/image001.jpg}
        \caption{Hinweisetikett zum Abklemmen der Batterie während der Installation.}
    \end{subfigure}\hfill
    \begin{subfigure}{0.46\textwidth}
        \includegraphics[width=\linewidth]{digifiz_manual/image002.jpg}
        \caption{Warnhinweis am Sensorkabelbaum gegen das Einspeisen externer Spannungen.}
    \end{subfigure}
    \caption{Sicherheitsaufkleber, die dem Kabelsatz beiliegen.}
\end{figure}
