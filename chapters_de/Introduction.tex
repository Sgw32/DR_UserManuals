\chapter{Einleitung}\label{ch:introduction}

Dieses Benutzerhandbuch behandelt die digitalen Kombiinstrumente \ReplicaGenOne{} und \ReplicaNextLong{} für Fahrzeuge der Baureihen Volkswagen Golf~II/III, Jetta~II, Vento und Scirocco~II. Es fasst die Hardwarevarianten zusammen, beschreibt deren Funktionen und erklärt, wie die Armaturen installiert, konfiguriert, betrieben, gelagert und gewartet werden. Die Anleitung richtet sich an Fahrzeughalter, Kfz-Elektriker und Servicezentren, die das Produkt nachrüsten. Die erweiterte Ausgabe beschreibt außerdem die Golf-Mk3- und Volkswagen-Vento-Armaturen, deren Verdrahtung und Kalibrierung den früheren Plattformen entspricht.

Die folgenden Kapitel stellen das Kennzeichnungsschema der Produkte, die Belegungen der Steckverbinder, die Betriebsbedingungen sowie ausführliche Installations- und Konfigurationsverfahren vor. Außerdem enthält das Handbuch Wartungs- und Fehlersuchhinweise für beide \Replica{}-Generationen, sodass das Kombiinstrument auch ohne Werksdokumentation instand gesetzt werden kann.

\begin{figure}[htbp]
    \centering
    \begin{subfigure}{0.48\textwidth}
        \includegraphics[width=\linewidth]{digifiz_manual/image004.jpg}
        \caption{Lieferumfang für die Konfiguration GART~8--MGF.}
    \end{subfigure}\hfill
    \begin{subfigure}{0.48\textwidth}
        \includegraphics[width=\linewidth]{digifiz_manual/image005.jpg}
        \caption{Typischer Inhalt eines GART-Pakets.}
    \end{subfigure}
    \begin{subfigure}{0.48\textwidth}
        \includegraphics[width=\linewidth]{digifiz_manual/image006.jpg}
        \caption{Rückansicht der GACS-Ausführung mit Einzelstecker.}
    \end{subfigure}\hfill
    \begin{subfigure}{0.48\textwidth}
        \includegraphics[width=\linewidth]{digifiz_manual/image007.jpg}
        \caption{Rückansicht der GACT-Ausführung mit Doppelstecker.}
    \end{subfigure}
    \caption{Repräsentative \ReplicaGenOne{}- und \ReplicaNextLong{}-Kombiinstrumente, die mit diesem Handbuch geliefert werden.}
\end{figure}

Jede Variante wird mit den Komponenten geliefert, die für den vorgesehenen Antriebsstrang, die Messeinheiten und den Kabelbaum benötigt werden. In späteren Kapiteln werden die Variantenkennzeichnungen entschlüsselt und Steckverbindertabellen bereitgestellt, damit das Kombiinstrument sicher eingebunden werden kann.
