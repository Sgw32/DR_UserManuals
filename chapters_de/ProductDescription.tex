\chapter{Beschreibung und Betrieb des Produkts}\label{ch:description}

\section{Einsatzzweck}
Die \ReplicaGenOne{}- und \ReplicaNextLong{}-Kombiinstrumente ersetzen die ursprünglichen Volkswagen-Armaturenträger und erweitern deren Funktionsumfang. Sie bieten digitale Anzeigen für Geschwindigkeit, Motordrehzahl, Kühlmitteltemperatur, Tankinhalt und MFA-Auswertungen und unterstützen sowohl mechanische Tachowellen als auch elektronische Geschwindigkeitssensoren. \ReplicaGenOneShort{}-Einheiten integrieren einen Bluetooth-Controller, während \ReplicaNextShort{} ein Wi-Fi-basiertes Konfigurationsmodul und optionale Erweiterungskomponenten ergänzt.

\section{Modellkennzeichnung}
Jedes Kombiinstrument ist mit einem vierstelligen Code markiert, der Antriebsart, Fertigungsform, Geschwindigkeitssensor-Schnittstelle und Kabelbaumgeneration beschreibt. Optionale Ziffern geben den unterstützten Drehzahlmesserbereich an, und ein zusätzlicher dreistelliger Suffix kennzeichnet Exportmessgrößen.

\subsection{Vierstellige Kennung}
\begin{description}
    \item[Position~1] \textbf{G} für Ottomotoren oder \textbf{D} für Dieselmotoren.
    \item[Position~2] \textbf{A} für werksseitig montierte Einheiten oder \textbf{M} für Selbstbausätze.
    \item[Position~3] \textbf{C} für einen mechanischen Tachowellenantrieb oder \textbf{R} für einen elektronischen Geschwindigkeitssensor.
    \item[Position~4] \textbf{T} für den Vorfacelift-Kabelbaum (CE~1) oder \textbf{S} für den Facelift-Kabelbaum (CE~2).
\end{description}
Eine nachgestellte Ziffer gibt die maximal angezeigte Motordrehzahl in Tausend U/min an (z.~B. steht die „8“ eines GACT8-Kombiinstruments für eine 8000-U/min-Skala).

\subsection{Messgrößensuffix}
Exportvarianten können einen dreistelligen Suffix aus der Menge \texttt{MGFK} tragen:
\begin{description}
    \item[M] Meilen pro Stunde,
    \item[G] Gallonen,
    \item[F] Fahrenheit,
    \item[K] Kelvin.
\end{description}
Ein \texttt{GART8-MGF}-Kombiinstrument ist demnach ein benzinbetriebenes, werksmontiertes CE~2-Gerät mit elektronischem Geschwindigkeitssensor, 8000-U/min-Drehzahlmesser und imperialen Maßeinheiten.

\section{Modellpalette}
{\scriptsize
\begin{tblr}{
    colspec={Q[l,2.2cm] X[l]},
    hlines
}
\textbf{Modell} & \textbf{Beschreibung} \\
GACT & Benziner, fertig montiert, Tachowellenantrieb, zwei Stecker, 7000-U/min-Skala. \\
GART & Benziner, fertig montiert, externer elektronischer Geschwindigkeitssensor, zwei Stecker, 7000-U/min-Skala. \\
GAC & Benziner, fertig montiert, Tachowellenantrieb, Einzelstecker, 7000-U/min-Skala. \\
GARS & Benziner, fertig montiert, externer elektronischer Geschwindigkeitssensor, Einzelstecker, 7000-U/min-Skala. \\
GACT8 & Benziner, fertig montiert, Tachowellenantrieb, zwei Stecker, 8000-U/min-Skala. \\
GART8 & Benziner, fertig montiert, externer elektronischer Geschwindigkeitssensor, zwei Stecker, 8000-U/min-Skala. \\
GACS8 & Benziner, fertig montiert, Tachowellenantrieb, Einzelstecker, 8000-U/min-Skala. \\
GARS8 & Benziner, fertig montiert, externer elektronischer Geschwindigkeitssensor, Einzelstecker, 8000-U/min-Skala. \\
DACT & Diesel, fertig montiert, Tachowellenantrieb, zwei Stecker, 6000-U/min-Skala. \\
DART & Diesel, fertig montiert, externer elektronischer Geschwindigkeitssensor, zwei Stecker, 6000-U/min-Skala. \\
DACS & Diesel, fertig montiert, Tachowellenantrieb, Einzelstecker, 6000-U/min-Skala. \\
DARS & Diesel, fertig montiert, externer elektronischer Geschwindigkeitssensor, Einzelstecker, 6000-U/min-Skala. \\
MT & Selbstbausatz mit zwei Steckern. \\
M.S. & Selbstbausatz mit Einzelstecker. \\
NEXT-GART & \ReplicaNextLong{}, 8000-U/min-Skala, zwei Stecker, elektronischer Geschwindigkeitssensor. \\
NEXT-GARS & \ReplicaNextLong{}, 8000-U/min-Skala, Einzelstecker, elektronischer Geschwindigkeitssensor. \\
NEXT-MT & \ReplicaNextLong{}-Selbstbausatz mit zwei Steckern. \\
NEXT-MS & \ReplicaNextLong{}-Selbstbausatz mit Einzelstecker. \\
\end{tblr}}

\section{Steckerbelegungen}
\subsection{Kombiinstrumente mit zwei Steckern}
\begin{figure}[htbp]
    \centering
    \includegraphics[width=0.72\textwidth]{digifiz_manual/image008.jpg}
    \caption{Steckeranordnung für \ReplicaGenOne{}-Kombiinstrumente mit zwei Steckverbindern.}
\end{figure}

\noindent\textbf{Weißer Stecker}

{\scriptsize
\begin{tblr}{
    colspec={Q[l,1.4cm] X[l]},
    hlines
}
\textbf{Pin} & \textbf{Belegung} \\
1 & Blinker-Ausgang, gegen Masse geschaltet für die Kontrollleuchte. \\
2 & Frei --- nicht belegt. \\
3 & Klemme~58, positive Versorgung der Hintergrundbeleuchtung. \\
4 & Eingang für den resistiven Kühlmitteltemperatursensor. \\
5 & Eingang für den resistiven Tankgeber. \\
6 & Masse. \\
7 & Zusätzliche Masse. \\
8 & Klemme~1, Motordrehzahlsignal (Zündspule, Verteiler oder anderes Signal bis 12~V mit möglichen 300-V-Spitzen). \\
9 & MFA-Modusleitung zum Wechseln der MFA-Funktionen. \\
10 & UNR-Dauerplus (bei \ReplicaGenOneShort{} ungenutzt, Hauptversorgung bei \ReplicaNextShort{}). \\
11 & MFA-Temperaturleitung „+“ für den Außentemperatursensor (\ReplicaNextShort{}). \\
12 & MFA-Öltemperaturleitung (nur \ReplicaNextShort{}). \\
13 & Klemme~56a, Eingang Fernlichtkontrollleuchte (+12~V aktiv). \\
\end{tblr}}

\noindent\textbf{Schwarzer Stecker}

{\scriptsize
\begin{tblr}{
    colspec={Q[l,1.4cm] X[l]},
    hlines
}
\textbf{Pin} & \textbf{Belegung} \\
1 & Klemme~15, geschaltetes +12~V vom Zündschloss. \\
2--4 & Nicht belegt. \\
5 & Eingang Handbremskontrollleuchte (Low-aktiv). \\
6 & Klemme~61, Generatorwarnleuchte mit 120~\ensuremath{\Omega}-Erregerwiderstand. \\
7 & Öldruckschalter, 0{,}3~bar. \\
8 & Öldruckschalter, 1{,}8~bar. \\
9 & Nicht belegt. \\
10 & Eingang Vorglühkontrollleuchte (+12~V aktiv, nur Diesel). \\
11 & Hall-Sensoreingang für optionale Geschwindigkeitssensoren. \\
12 & Leitung zur Auswahl des MFA-Speicherblocks. \\
13 & Leitung zum Zurücksetzen der MFA. \\
\end{tblr}}

\subsection{Kombiinstrumente mit Einzelstecker}
Armaturen mit Einzelstecker verwenden die in \autoref{fig:single-connector} dargestellte Zuordnung. Der Kabelbaum führt dieselben Signale wie die Varianten mit zwei Steckern zusammen und bündelt sie in einem Anschluss.

\begin{figure}[htbp]
    \centering
    \includegraphics[width=0.65\textwidth]{digifiz_manual/image009.png}
    \caption{Einzelstecker-Layout für kompakte \Replica{}-Kombiinstrumente.}
    \label{fig:single-connector}
\end{figure}

\subsection{Prospektiver Scirocco-/Passat-Kabelbaum}
Der vorgesehene Scirocco-/Passat-Kabelbaum nutzt zwei Stecker. Deren Funktionen sind nachfolgend zusammengefasst.

\begin{samepage}
\noindent\textbf{5-poliger Stecker}
{\scriptsize
\begin{tblr}{
    colspec={Q[l,2.6cm] X[l]},
    hlines
}
\textbf{Pin} & \textbf{Belegung} \\
1~(D3) & Kontakt für Automatikstellung „D“. Legt die Anzeigelampe auf Masse, wenn der Wählhebel in Position~D steht. \\
2~(D2) & Kontakt für Automatikstellung „2“. Legt die „2“-Lampe auf Masse, wenn der Wählhebel in Position~2 steht. \\
3~(D1) & Kontakt für Automatikstellung „1“. Legt die „1“-Lampe auf Masse, wenn der Wählhebel in Position~1 steht. \\
4~(SA) & Gemeinsame Versorgung der Automatik-Anzeige (\emph{Schaltanzeige}); stellt +12~V für die Ganganzeigen bereit. \\
5~(SPERRE) & Anlasssperre vom Wählhebel. Geschlossen in „P“ oder „N“, um das Starten zu erlauben. \\
\end{tblr}}
\end{samepage}

\begin{samepage}
\noindent\textbf{14-poliger Stecker}
{\scriptsize
\begin{tblr}{
    colspec={Q[l,2.6cm] X[l]},
    hlines
}
\textbf{Pin} & \textbf{Belegung} \\
1~(KL~58) & Versorgung der Hintergrundbeleuchtung. \\
2~(MASS) & Masseanschluss. \\
3~(TANK) & Eingang Tankgeber. \\
4~(TEMP) & Eingang Kühlmitteltemperaturgeber. \\
5~(KL~1) & Motordrehzahlsignal (Klemme~1). \\
6~(UHR) & Dauerplus für Uhr und Speicher. \\
7~(FERNL) & Eingang Fernlichtkontrollleuchte. \\
8~(reserved) & Nicht belegt. \\
9~(OEL~1.8) & Öldruckschalter, 1{,}8~bar. \\
10~(CAT~VORGL(-)) & Eingang Katalysatorvorwärmung / Diesel-Vorglühlampe (Low-aktiv). \\
11~(OEL~0.3) & Öldruckschalter, 0{,}3~bar. \\
12~(KL~61) & Generatorwarnlampe und Erregerspannung. \\
13~(KL~49a) & Gemeinsame Blinkerkontrollleuchte. \\
14~(KL~15) & Zündungsplus. \\
\end{tblr}}
\end{samepage}

\subsection{Belegung für Golf~Mk1}
Volkswagen-Fahrzeuge der ersten Generation verwenden folgende Anschlüsse:
\begin{enumerate}
    \item Versorgung für Beleuchtung und Abblendlicht.
    \item MASSE~31 als Massebezug.
    \item TANK-Anschluss des Tankgebers.
    \item TEMP-Anschluss des Kühlmittelgebers.
    \item KL~1 als Drehzahlsignal.
    \item UHR als Dauerplus.
    \item KL~56 für das Fernlichtsignal.
    \item OIL (HIGH) --- Öldruckschalter 1,8~bar.
    \item OIL (LOW) --- Öldruckschalter 0,3~bar.
    \item Diesel-Vorglühkontrollleuchte.
    \item CHOKE-Eingang (ungenutzt).
    \item KL~61 Generatorlampe.
    \item Blinkereingang (kombiniert links/rechts).
    \item KL~15 Zündungsversorgung.
\end{enumerate}
\begin{figure}[htbp]
    \centering
    \includegraphics[width=0.75\textwidth]{digifiz_manual/image010.png}
    \caption{Anschlussplan des Kabelbaums für Mk1-Installationen.}
\end{figure}

\subsection{Pinbelegung für \ReplicaNextShort{} im Golf~Mk3}
Der speziell für den Golf~Mk3 vorgesehene \ReplicaNextShort{}-Kabelbaum endet in einem 28-poligen Stecker. Seine Zuordnungen
folgen dem originalen Volkswagen-Digifiz-Mk3-Schaltplan, sodass der Kabelbaum direkt an die Fahrzeugverkabelung oder passende
Adapter angeschlossen werden kann (siehe \autoref{tab:mk3-pinout-de}).

\begin{table}[htbp]
    \centering
    {\scriptsize
    \begin{tblr}{
        colspec={Q[c,0.09\linewidth] Q[l,0.81\linewidth]},
        hlines,
        row{1} = {font=\bfseries},
    }
    Pin & Belegung \\
    1 & Außentemperaturfühler~\mbox{-G17-}, Masse (GND)\textsuperscript{$\dagger$}. \\
    2 & Kühlmittelmangelgeber~\mbox{-G32-}. \\
    3 & Klemme~31, Masse. \\
    4 & MFA-Speichertaste~\mbox{-E109-}, Reset-Leitung\textsuperscript{$\dagger$}. \\
    5 & Klemme~31, Masse. \\
    6 & MFA-Speichertaste~\mbox{-E109-}, Memory-Leitung\textsuperscript{$\dagger$}. \\
    7 & Geschwindigkeitssensor-Ausgang (VSS). \\
    8 & Öldruckschalter~\mbox{-F1-}, \SI{1.8}{bar}. \\
    9 & Öldruckschalter~\mbox{-F22-}, \SI{0.3}{bar}. \\
    10 & Klemme~1 / Klemme~W, Drehzahlsignal. \\
    11 & Klemme~30, Batteriespannung (B+). \\
    12 & Klemme~58b, Beleuchtung. \\
    13 & Klemme~15, geschaltetes B+. \\
    14 & Ohne Kontakt. \\
    15 & MFA-Moduswahltaste~\mbox{-E86-}. \\
    16 & Generator-Kontrollleuchte~\mbox{-K2-}, Klemme~61. \\
    17 & Motoröltemperaturfühler~\mbox{-G8-}\textsuperscript{$\dagger$}. \\
    18 & Brems- und Handbremskontrollleuchte~\mbox{-K7-}. \\
    19 & Außentemperaturfühler~\mbox{-G17-}, Signal\textsuperscript{$\dagger$}. \\
    20 & Motorkontrollleuchte~\mbox{-K83-} oder Diesel-Glühwendelanzeige~\mbox{-K29-}. \\
    21 & Tankgeber / Kraftstoffanzeige~\mbox{-G1-}. \\
    22 & Linke Blinker-Kontrollleuchte~\mbox{-K65-}. \\
    23 & Kühlmitteltemperaturanzeige~\mbox{-G3-}. \\
    24 & Rechte Blinker-Kontrollleuchte~\mbox{-K94-}. \\
    25 & Fernlichtkontrollleuchte~\mbox{-K1-}. \\
    26 & Verbrauchssignal-Eingang\textsuperscript{$\dagger$}. \\
    27 & Fahrgeschwindigkeitssignal vom Tachosensor~\mbox{-G22-}. \\
    28 & Automatikgetriebe-Anzeige (AG4) für die Wählhebelstellung. \\
    \end{tblr}}
    \caption{Pinbelegung des Mk3-\ReplicaNextShort{}-Kabelbaums.}
    \label{tab:mk3-pinout-de}
\end{table}

Mit \textsuperscript{$\dagger$} markierte Leitungen verlassen den 28-poligen Stecker über die mitgelieferten JST-Abgriffe. Sie
führen verdrillte Paare für die Außen- und Öltemperatursensoren sowie direkte Anschlüsse an die \textsc{MFA}-Tasten und das
Verbrauchssignal.

\subsection{Service-Steckverbinder auf der Leiterplatte}
Der dritte Steckverbinder auf der Leiterplatte spiegelt die Armaturenstecker wider; die Pins sind bei \ReplicaGenOneShort{}- und \ReplicaNextShort{}-Geräten von rechts nach links nummeriert. Er dient als Service-Schnittstelle mit den in \autoref{tab:service-connector} aufgeführten Belegungen.

\begin{table}[htbp]
    \centering
    \caption{Belegung des Service-Steckverbinders.}
    \label{tab:service-connector}
    {\scriptsize
    \begin{tblr}{
        colspec={Q[l,1.9cm] X[l]},
        hlines,
    }
        \textbf{Position} & \textbf{Belegung} \\
        1 & Kontrollausgang. \\
        2 & Geschwindigkeitssensor-Eingang (SPM\_M). \\
        3 & Fahrzeugmasse. \\
        4 & Kontrollausgang. \\
        5 & Linker Blinker, Optokoppler-Eingang. \\
        6 & Rechter Blinker, Optokoppler-Eingang. \\
        7 & Zündungsplus +12~V. \\
        8 & Eingang für Diesel-spezifische Funktionen. \\
        9 & Kontrollleuchteneingang (positiv). \\
        10 & Alternativer Drehzahleingang (ungenutzt, nur \ReplicaNextShort{}). \\
        11 & \ReplicaGenOneShort{}: Kontrollausgang (normal offen); \ReplicaNextShort{}: Bremssignal (Low-aktiv). \\
        12 & Reserviert. \\
        13 & Check-Engine-Eingang. \\
        14 & Ohne Kontakt. \\
    \end{tblr}}
\end{table}

\subsection{Zusätzliche Erweiterungsstecker}
Auf der Hauptplatine sind drei zusätzliche Vierpolbuchsen vorgesehen, um Kabelbaum-Upgrades und Servicearbeiten zu erleichtern:
\begin{itemize}
    \item \textbf{Analoge Erweiterungssignale:} dedizierter Abgriff für zusätzliche analoge Eingänge bei der Integration kundenspezifischer Sensoren.
    \item \textbf{MFA-Spiegel:} Dupliziert den Standard-\textsc{MFA}-Stecker, um die Bordcomputer-Signale parallel auszuwerten.
    \item \textbf{Analoge Duplikate:} Führt die Eingänge für Öltemperatur, Außentemperatur und Bremssignal doppelt heraus, damit sie an externe Logger oder Überwachungsmodule weitergegeben werden können.
\end{itemize}
Alle drei Steckplätze verwenden den \mbox{KF2510-4p}-Gegenstecker, der dem Bausatz nicht beiliegt und bei Bedarf separat beschafft werden muss.

\begin{figure}[htbp]
    \centering
    \includegraphics[width=0.6\textwidth]{digifiz_manual/ext_conn.png}
    \caption{Anordnung der Zusatzstecker auf der Hauptplatine.}
\end{figure}

\begin{table}[htbp]
    \centering
    {\small
    \begin{tblr}{
        colspec={Q[l,2.3cm] Q[c,1.3cm] X[l]},
        hlines,
        row{1} = {font=\bfseries}
    }
    Stecker & Pin & Belegung \\
    Stecker~I & 4 & Analoger Zusatz-Eingang~1 \\
    Stecker~I & 3 & Masse (GND) \\
    Stecker~I & 2 & Analoger Zusatz-Eingang~2 \\
    Stecker~I & 1 & VCC (3V3, ungesichert\textbf{!!!}) \\
    Stecker~II & 4 & MFA-Reset \\
    Stecker~II & 3 & Masse (GND) \\
    Stecker~II & 2 & MFA-Speicherblock \\
    Stecker~II & 1 & MFA-Modus \\
    Stecker~III & 4 & Ausgang Öltemperatursensor \\
    Stecker~III & 3 & Masse (GND) \\
    Stecker~III & 2 & Ausgang Außentemperatursensor \\
    Stecker~III & 1 & Bremssignal \\
    \end{tblr}}
    \caption{Pinbelegung der zusätzlichen Erweiterungsstecker.}
\end{table}

\section{Software und Lieferumfang}
Die Firmware des Kombiinstruments ist unter folgender Adresse veröffentlicht:
\displayurl{https://github.com/Sgw32/DigifizReplica}
Es stehen zwei Lieferumfänge zur Verfügung:
\begin{itemize}
    \item \textbf{\ReplicaGenOne{}:} Komplettes Kombiinstrument, Kabelbaum für Außen- und Öltemperatursensor, USBasp-Programmiergerät und (bei Remote-Sensoren) ein Geschwindigkeitssensor-Kabelbaum.
    \item \textbf{\ReplicaNextLong{}:} Komplettes Kombiinstrument sowie ein Kabelbaum für den elektronischen Geschwindigkeitssensor.
\end{itemize}
