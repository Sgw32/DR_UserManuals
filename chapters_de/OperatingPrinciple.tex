\chapter{Funktionsprinzip} \label{ch:operating-principle}

\ReplicaGenOne{}-Kombiinstrumente verwenden das originale Volkswagen-Gehäuse, die werkseitigen CE~1- bzw. CE~2-Steckverbinder sowie entweder den mechanischen Tachowellenanschluss oder einen elektronischen Geschwindigkeitssensor wieder.
Die Hauptplatinen der \ReplicaGenOneShort{} basieren auf einer Leiterplatte aus Glasfaser, die mit diskreten Bauteilen bestückt ist und von einem ATmega~2560-Mikrocontroller sowie MAX~7219-Anzeigetreibern angesteuert wird.

\ReplicaNextLong{} setzt auf ein ESP32-S3-System-on-Chip und führt ein neu gefertigtes, SLA-gedrucktes Gehäuse, eine überarbeitete Frontplatte und Abdeckung sowie eine Adapterplatine für die Steckverbinder ein.
Das Display der \ReplicaNextShort{} wird von WS2812-Adress-LEDs beleuchtet, die hinter dem Frontrahmen montiert sind, und der zugehörige Kabelbaum enthält standardmäßig den elektronischen Geschwindigkeitssensor.

Beide Generationen teilen sich das gleiche Displaylayout und die MFA-Seiten, sodass Installationsabläufe und der tägliche Betrieb zwischen den Hardware-Revisionen vertraut bleiben.
