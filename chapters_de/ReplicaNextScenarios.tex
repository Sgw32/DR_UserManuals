\chapter{Typische Konfigurationsszenarien für die \ReplicaNextShort{}}\label{ch:replica-next-scenarios}

\begin{description}
    \item[Hotspot nicht sichtbar] Gehen Sie näher an das Fahrzeug und stellen Sie sicher, dass es im Freien steht. Deaktivieren Sie mobile Daten, löschen Sie veraltete WLAN-Profile und verbinden Sie sich erneut mit \texttt{Digifiz\_AP} (oder \texttt{PHOL-LABS2}).
    \item[404 unter \texttt{192.168.4.1}] Schalten Sie mobile Daten auf dem Telefon oder Laptop aus und laden Sie die Seite neu. Die Captive-Portal-Erkennung von Android/iOS stört häufig, bis das Mobilfunkmodem deaktiviert ist.
    \item[Firmware-Updates] Öffnen Sie den Reiter \emph{WiFi} und wählen Sie die mitgelieferte Datei \texttt{Digifiz.bin}. Die aktuellen Versionen sind unter folgendem Link veröffentlicht.
        \displayurl{https://github.com/Sgw32/DigifizReplica/releases}
        Klicken Sie auf \emph{Upload}. Der erste Versuch kann fehlschlagen; wiederholen Sie den Upload gegebenenfalls. Erfolgreiche Flashvorgänge leiten auf eine Bestätigungsseite weiter. Notieren Sie vor dem Update den Kilometerstand und stellen Sie ihn anschließend mit \verb|11 <mileage>| wieder her.
    \item[Befehle werden ignoriert] Aktualisieren Sie den Browser, kehren Sie zum Reiter \emph{Control} zurück und senden Sie den Befehl erneut. Drücken Sie nach der Eingabe unbedingt auf \emph{Process}.
    \item[Geschwindigkeitsanzeige fehlerhaft] Verbinden Sie sich per WLAN, fahren Sie laut Anzeige \SI{100}{\kilo\metre\per\hour}, notieren Sie die GPS-Geschwindigkeit und senden Sie \verb|1 <gps_value>| (z.~B. \verb|1 85|), um \paramname{PARAMETER\_SPEEDCOEFFICIENT} auf den verifizierten Wert zu setzen.
    \item[Drehzahlanzeige fehlerhaft] Passen Sie \paramname{PARAMETER\_RPMCOEFFICIENT} an. Ältere Firmware nutzt \verb|0 <value>|, aktuelle Versionen \verb|22 <value>|. Beispiel: \verb|22 1500| halbiert die Anzeige gegenüber \verb|22 3000|.
    \item[Display zu dunkel] Deaktivieren Sie die automatische Helligkeit mit \verb|13 0| und erhöhen Sie anschließend den manuellen Wert (z.~B. \verb|14 50|). Experimentieren Sie mit Werten zwischen 45 und 55; vermeiden Sie Werte über 60, um die LED-Lebensdauer zu erhalten.
    \item[Uhr einstellen] Nutzen Sie das Webterminal (oder bei älteren Builds Serial Bluetooth Terminal), um \verb|255 <hours>| gefolgt von \verb|254 <minutes>| zu senden. Beispiel: \verb|255 23| und \verb|254 55| stellt 23:55 ein.
    \item[Tankanzeige bleibt stehen] Klemmen Sie die Batterie ab und messen Sie den Widerstand zwischen Tankgeber-Pin und Fahrzeugmasse. Gültige Werte liegen typischerweise zwischen \SI{30}{} und \SI{300}{\ohm}. Beheben Sie Kurzschlüsse unter \SI{5}{\ohm} oder Unterbrechungen, bevor Sie wieder anschließen. Variieren die Messwerte korrekt, zeigt die Skala jedoch nichts an, protokollieren Sie \verb|adc 0|-Werte bei mehreren Tankständen und teilen Sie sie PHOL-LABS Kft mit.
    \item[Kraftstoffdurchfluss ungenau] Der optionale Durchflusssensor liefert emulierte Daten und ist ohne Saugrohrdrucksensor unzuverlässig. Betrachten Sie die Anzeigen als experimentell.
    \item[Kühlmitteltemperatur außerhalb des Bereichs] Justieren Sie \paramname{PARAMETER\_COOLANT\_MIN\_R} und \paramname{PARAMETER\_COOLANT\_MAX\_R}. Beispiel: \verb|27 30| senkt die Schwelle „1~bar“ auf \SI{30}{\celsius}.
    \item[Öl- oder Außentemperatur fehlt] Messen Sie bei abgeklemmter Batterie und kaltem Motor den Sensorwiderstand. Ölsensoren sollten etwa \SI{2}{\kilo\ohm} \ensuremath{\pm}\SI{0.3}{\kilo\ohm} anzeigen, Außensensoren etwa \SI{10}{\kilo\ohm} \ensuremath{\pm}\SI{2}{\kilo\ohm}. Passen Sie \paramname{PARAMETER\_NORMAL\_RESISTANCE\_OIL} (Befehl~20) oder \paramname{PARAMETER\_NORMAL\_RESISTANCE\_AMB} (Befehl~21) an; niedrigere Werte senken die angezeigte Temperatur, höhere Werte erhöhen sie. Anhaltende Probleme sollten mittels \verb|adc 0|-Protokollen analysiert und mit PHOL-LABS Kft geklärt werden.
    \item[Änderung der Interface-Farbe] Nutzen Sie die Befehle 31--33 zur Einstellung der RGB-Werte. Neuere Firmware-Versionen bieten zudem grafische Farbauswahl im Webinterface, daher empfiehlt sich ein regelmäßiges Update.
\end{description}
