\chapter{Einrichtung und Wartung der \ReplicaNextLong{}}\label{ch:replica-next-setup}

Dieses Kapitel bezieht sich auf das in \autoref{fig:next-hardware} gezeigte \ReplicaNextLong{}-Kombiinstrument.

\begin{figure}[htbp]
    \centering
    \includegraphics[width=0.6\textwidth]{digifiz_manual/image019.png}
    \caption{\ReplicaNextLong{}-Kombiinstrument.}
    \label{fig:next-hardware}
\end{figure}

\section{Umgang mit dem Panel}
\begin{itemize}
    \item Die UV-bedruckte Polycarbonat-Frontplatte muss vor Kratzern und Fremdkörpern geschützt werden. Größere Beschädigungen gelten nicht als Garantiefall und erfordern Ersatzteile von PHOL-LABS Kft.
    \item Die Echtzeituhr wird über das Wi-Fi-Bedienfeld konfiguriert. Sie setzt sich zurück, sobald die Dauerplusversorgung getrennt wird.
\end{itemize}

\section{Wi-Fi-Steuerportal}
Konfiguration, Datenerfassung und Firmwarepflege erfolgen über die integrierte Webanwendung.
\begin{itemize}
    \item Verbinden Sie sich mit dem WLAN-Zugangspunkt des Kombiinstruments. Deaktivieren Sie mobile Daten und stellen Sie eine Verbindung zu \texttt{Digifiz\_AP} (Passwort \texttt{87654321}) her; einige Revisionen senden als \texttt{PHOL-LABS2} mit demselben Passwort.
    \item Die Standard-IP-Adresse lautet \texttt{192.168.4.1}. Ist das Kombiinstrument in ein anderes Netzwerk eingebunden, scannen Sie das Subnetz mit einem IP-Tool nach einer Adresse, die auf \texttt{.32} endet.
    \item Das Portal umfasst die Reiter \emph{WiFi}, \emph{Control}, \emph{Settings}, \emph{Colors} und \emph{About} (\autoref{fig:next-control-tabs}). Im WiFi-Reiter werden Netzwerke konfiguriert und Firmwaredateien hochgeladen; der Control-Reiter passt Betriebsparameter an; Settings stellt einen strukturierten Editor für sämtliche Firmwareparameter bereit; Colors verwaltet mehrsegmentige Farbschemata; About führt Autoren- und Versionsinformationen auf.
\end{itemize}

\begin{figure}[htbp]
    \centering
    \begin{subfigure}{0.48\textwidth}
        \includegraphics[width=\linewidth]{digifiz_manual/image020.png}
        \caption{Übersicht des Control-Reiters.}
    \end{subfigure}\hfill
    \begin{subfigure}{0.48\textwidth}
        \includegraphics[width=\linewidth]{digifiz_manual/image021.png}
        \caption{Nummerierte Bedienelemente und Eingabefelder.}
    \end{subfigure}
    \caption{Wi-Fi-Steueroberfläche der \ReplicaNextShort{}.}
    \label{fig:next-control-tabs}
\end{figure}

\section{Befehle eingeben}
Der Reiter \emph{Control} stellt eine Befehlszeile (1), die Schaltfläche \emph{Process} (2), ein Ergebnisfenster (3), Schnellbedienfelder (4) sowie die Schaltflächen \emph{Save} (5) und \emph{Reset} (6) bereit. Geben Sie Befehle als Leerzeichen-getrennte Paare \verb|<Nummer> <Wert>| ein und verwenden Sie ausschließlich ganze Zahlen; Satzzeichen oder Anführungszeichen sind nicht nötig. \autoref{fig:next-command-example} zeigt das Interface beim Umschalten der automatischen Helligkeit.

\begin{figure}[htbp]
    \centering
    \includegraphics[width=0.55\textwidth]{digifiz_manual/image022.png}
    \caption{Beispielsequenz zum Deaktivieren der automatischen Helligkeit.}
    \label{fig:next-command-example}
\end{figure}

\section{Befehlsreferenz}
\begin{table}[htbp]
    \centering
    \caption{Wichtigste Konfigurationsbefehle der \ReplicaNextShort{}.}
    \label{tbl:next-commands}
    {\scriptsize
    \begin{tblr}{
        colspec = {Q[c,0.14\linewidth] Q[l,0.34\linewidth] Q[l,0.5\linewidth]},
        rowsep = 2pt,
    }
        \toprule
        \textbf{Befehl} & \textbf{Name} & \textbf{Beschreibung} \\
        \midrule
        22 (oder 0) & \paramname{PARAMETER\_RPMCOEFFICIENT} & Kalibrierfaktor der Motordrehzahl (100--10000). \\
        1  & \paramname{PARAMETER\_SPEEDCOEFFICIENT} & Kalibrierfaktor für die Geschwindigkeit (10--255). \\
        2  & \paramname{PARAMETER\_COOLANTTHERMISTORB} & Beta-Koeffizient des Kühlmittelthermistors (2000--5000). \\
        3  & \paramname{PARAMETER\_OILTHERMISTORB} & Beta-Koeffizient des Ölthermistors (2000--5000). \\
        4  & \paramname{PARAMETER\_AIRTHERMISTORB} & Beta-Koeffizient des Außentemperaturthermistors (2000--5000). \\
        5  & \paramname{PARAMETER\_TANKMINRESISTANCE} & Minimaler Tankgeberwiderstand (0--1000~\ohm). \\
        6  & \paramname{PARAMETER\_TANKMAXRESISTANCE} & Maximaler Tankgeberwiderstand (100--1000~\ohm). \\
        7  & \paramname{PARAMETER\_TAU\_COOLANT} & Filterkonstante der Kühlmittelanzeige (1--50; höhere Werte reagieren schneller). \\
        8  & \paramname{PARAMETER\_TAU\_OIL} & Filterkonstante der Öltemperaturanzeige (1--50). \\
        9  & \paramname{PARAMETER\_TAU\_AIR} & Filterkonstante der Außentemperaturanzeige (1--50). \\
        10 & \paramname{PARAMETER\_TAU\_TANK} & Filterkonstante der Tankanzeige (1--50). \\
        11 & \paramname{PARAMETER\_MILEAGE} & Gesamtkilometerstand (0--999999). \\
        12 & \paramname{PARAMETER\_DAILY\_MILEAGE} & Tageskilometerzähler (0--9999). \\
        13 & \paramname{PARAMETER\_AUTO\_BRIGHTNESS} & Automatische Helligkeit aktivieren (1 = an, 0 = aus). \\
        14 & \paramname{PARAMETER\_BRIGHTNESS\_LEVEL} & Manuelle Helligkeit (0--60\%; Werte über 60 verkürzen die LED-Lebensdauer). \\
        15 & \paramname{PARAMETER\_TANK\_CAPACITY} & Tankvolumen in Litern (0--99; 55~L typisch für Golf~II/III und Vento). \\
        16 & \paramname{PARAMETER\_MFA\_STATE} & Aktiver MFA-Modus (normalerweise über Hardware-Eingang gesteuert). \\
        17 & \paramname{PARAMETER\_BUZZER\_OFF} & Summer deaktivieren (1 deaktiviert, 0 aktiviert; \ReplicaNextShort{} besitzt keinen Summer). \\
        18 & \paramname{PARAMETER\_MAX\_RPM} & Drehzahlmesserskala (typisch 8000, Bereich 4000--16000). \\
        19 & \paramname{PARAMETER\_NORMAL\_RESISTANCE\_COOLANT} & Sensorwiderstand bei \SI{25}{\celsius} (1000--10000~\ohm). \\
        20 & \paramname{PARAMETER\_NORMAL\_RESISTANCE\_OIL} & Öltemperatursensor bei \SI{25}{\celsius} (1000--10000~\ohm). \\
        21 & \paramname{PARAMETER\_NORMAL\_RESISTANCE\_AMB} & Außentemperatursensor bei \SI{25}{\celsius} (1000--10000~\ohm). \\
        23 & \paramname{PARAMETER\_DOT\_OFF} & Verhalten des Uhr-Doppelpunkts (0 = blinkend, 1 = dauerhaft). \\
        24 & \paramname{PARAMETER\_BACKLIGHT\_ON} & Hintergrundbeleuchtung mit Abblendlicht aktivieren (bei \ReplicaNextShort{} ungenutzt). \\
        25 & \paramname{PARAMETER\_M\_D\_FILTER} & Medianfilter-Konstante (Legacy, normalerweise ungenutzt). \\
        26 & \paramname{PARAMETER\_COOLANT\_MAX\_R} & Schwelle für die Kühlmittel-Vollskala (\SI{100}{\celsius}--\SI{150}{\celsius}). \\
        27 & \paramname{PARAMETER\_COOLANT\_MIN\_R} & Schwelle für die „1~bar“-Markierung (\SI{0}{\celsius}--\SI{80}{\celsius}). \\
        31 & \paramname{PARAMETER\_MAINCOLOR\_R} & Rotanteil der UI-Farbe (0--255). \\
        32 & \paramname{PARAMETER\_MAINCOLOR\_G} & Grünanteil der UI-Farbe (0--255). \\
        33 & \paramname{PARAMETER\_MAINCOLOR\_B} & Blauanteil der UI-Farbe (0--255). \\
        37 & \paramname{PARAMETER\_RPM\_FILTER} & Reaktionsgeschwindigkeit des Drehzahlfilters (10--200, höhere Werte reagieren schneller). \\
        128 & \paramname{PARAMETER\_READ\_ADDITION} & Zum Befehl addieren, um den aktuellen Wert auszulesen. \\
        255 & \paramname{PARAMETER\_SET\_HOUR} & Uhrstunden setzen (24-Stunden-Format). \\
        254 & \paramname{PARAMETER\_SET\_MINUTE} & Uhrminuten setzen. \\
        253 & \paramname{PARAMETER\_RESET\_DAILY\_MILEAGE} & Tageskilometerzähler zurücksetzen. \\
        252 & \paramname{PARAMETER\_RESET\_DIGITAL} & Gespeicherte Parameter auf Werkseinstellungen setzen. \\
        \bottomrule
    \end{tblr}}
\end{table}

\section{Standardwerte}
\begin{table}[htbp]
    \centering
    \caption{Voreinstellungen der \ReplicaNextShort{}.}
    \label{tbl:next-defaults}
    {\scriptsize
    \begin{tblr}{
        colspec = {Q[l,0.42\linewidth] Q[c,0.15\linewidth] Q[l,0.43\linewidth]},
        rowsep = 2pt,
    }
        \toprule
        \textbf{Parameter} & \textbf{Standard} & \textbf{Hinweise} \\
        \midrule
        \paramname{PARAMETER\_RPMCOEFFICIENT} & 3000 & Typisch für Audi-Drehzahlsignale. \\
        \paramname{PARAMETER\_SPEEDCOEFFICIENT} & 100 & Auf \SI{100}{\kilo\metre\per\hour} kalibriert. \\
        \paramname{PARAMETER\_COOLANTTHERMISTORB} & 4000 &  \\
        \paramname{PARAMETER\_OILTHERMISTORB} & 4000 &  \\
        \paramname{PARAMETER\_AIRTHERMISTORB} & 3812 & Generation~2 nutzt 3600. \\
        \paramname{PARAMETER\_TANKMINRESISTANCE} & 35 & \ohm. \\
        \paramname{PARAMETER\_TANKMAXRESISTANCE} & 265 & \ohm. \\
        \paramname{PARAMETER\_TAU\_COOLANT} & 2 & Filterkonstante. \\
        \paramname{PARAMETER\_TAU\_OIL} & 2 & Filterkonstante. \\
        \paramname{PARAMETER\_TAU\_AIR} & 2 & Filterkonstante. \\
        \paramname{PARAMETER\_TAU\_TANK} & 2 & Filterkonstante. \\
        \paramname{PARAMETER\_MILEAGE} & Fahrzeugspezifisch & Beibehaltung des gespeicherten Kilometerstands. \\
        \paramname{PARAMETER\_DAILY\_MILEAGE} & 0 &  \\
        \paramname{PARAMETER\_AUTO\_BRIGHTNESS} & 1 & Aktiviert. \\
        \paramname{PARAMETER\_BRIGHTNESS\_LEVEL} & 25 & Standard für Generation~2; Generation~1/1.5 verwenden 7 bzw. 13. \\
        \paramname{PARAMETER\_TANK\_CAPACITY} & 63 & Liter. \\
        \paramname{PARAMETER\_MFA\_STATE} & 0 & Standard-MFA-Seite. \\
        \paramname{PARAMETER\_BUZZER\_OFF} & 1 & Summer deaktiviert. \\
        \paramname{PARAMETER\_MAX\_RPM} & 8000 & Drehzahlmesserskala. \\
        \paramname{PARAMETER\_NORMAL\_RESISTANCE\_COOLANT} & 1000 & \ohm{} bei \SI{25}{\celsius}. \\
        \paramname{PARAMETER\_NORMAL\_RESISTANCE\_OIL} & 1000 & \ohm{} bei \SI{25}{\celsius}. \\
        \paramname{PARAMETER\_NORMAL\_RESISTANCE\_AMB} & 2991 & 500~\ohm{} bei Sensoren der Generation~2. \\
        \paramname{PARAMETER\_DOT\_OFF} & 0 & Blinkender Uhr-Doppelpunkt. \\
        \paramname{PARAMETER\_BACKLIGHT\_ON} & 1 & Hintergrundbeleuchtung mit Abblendlicht aktiv. \\
        \paramname{PARAMETER\_M\_D\_FILTER} & 65535 & Legacy-Medianfilter. \\
        \paramname{PARAMETER\_COOLANT\_MAX\_R} & 120 & \si{\celsius}. \\
        \paramname{PARAMETER\_COOLANT\_MIN\_R} & 60 & \si{\celsius}. \\
        \paramname{PARAMETER\_MAINCOLOR\_R} & 180 & Standardton Gelbgrün. \\
        \paramname{PARAMETER\_MAINCOLOR\_G} & 240 & Standardton Gelbgrün. \\
        \paramname{PARAMETER\_MAINCOLOR\_B} & 6 & Standardton Gelbgrün. \\
        \paramname{PARAMETER\_RPM\_FILTER} & 70 & Filterreaktion. \\
        \paramname{PARAMETER\_UPTIME} & 0 & Laufzeitzähler. \\
        \bottomrule
    \end{tblr}}
\end{table}

\section{Parameter auslesen und Beispiele}
Zum Auslesen eines Parameters addieren Sie 128 zur Befehlsnummer (z.~B. meldet \verb|129 0| den Geschwindigkeitsfaktor). Typische Befehle sind das Deaktivieren der automatischen Helligkeit (\verb|13 0|), das erneute Aktivieren (\verb|13 1|), die Anpassung des Geschwindigkeitsfaktors (\verb|1 110| erhöht die Anzeige um 10~\%) und das Setzen des Kilometerstands (\verb|11 123456|). Die Uhr stellen Sie mit \verb|255 <hours>| und anschließend \verb|254 <minutes>| ein. Die Befehle~31--33 setzen die RGB-Komponenten der Benutzeroberfläche.

\section{Servicebefehle}
Aktuelle Firmwareversionen akzeptieren auch Klartextnamen, etwa \verb|PARAMETER_RPMCOEFFICIENT 3000|. Der Diagnosebefehl \verb|adc 0| gibt Rohwerte der Sensor-ADCs aus und unterstützt die Fehlersuche. Neue Firmwarestände bringen zusätzliche Farbeditoren mit sich, daher sollten Updates regelmäßig über den Reiter \emph{WiFi} eingespielt werden.

\section{Parametereditor im Reiter Settings}
Der Reiter \emph{Settings} spiegelt die Parameterlisten aus \autoref{tbl:next-commands} und \autoref{tbl:next-defaults} und ergänzt Metadaten zu Bereich, Beschreibung und Datentyp.
Nutzen Sie ihn, wenn Sie statt Befehlsnummern eine grafische Arbeitsweise bevorzugen.

\begin{enumerate}
    \item Klicken Sie auf \emph{Load Parameters}, um die aktuellen Werte aus dem Kombiinstrument zu laden. Der Browser zeigt jeden Eintrag mit Name, aktuellem Wert, Tooltip und Typ an.
    \item Geben Sie bei numerischen Parametern den gewünschten Wert in der Spalte \emph{New Value} ein. Das Interface erzwingt die zulässigen Bereiche aus den Spalten \emph{Min} und \emph{Max}. Boolean-Werte erscheinen als Kontrollkästchen.
    \item Drücken Sie \emph{Set}, um die Änderung sofort zu übertragen. Die Tabelle aktualisiert sich zur Bestätigung.
    \item Wiederholen Sie den Vorgang für alle gewünschten Parameter. Kehren Sie anschließend zum Reiter \emph{Control} zurück und klicken Sie auf \emph{Save parameters}, um die Konfiguration dauerhaft zu speichern.
\end{enumerate}

Für den Farbeditor müssen Sie zunächst die Firmware-Option für benutzerdefinierte Paletten aktivieren, bevor Sie zum Reiter \emph{Colors} wechseln.
Suchen Sie den booleschen Eintrag „Custom colour scheme“ (im Parameterfeld als \verb|PARAMETER_CUSTOM_COLORSCHEME_ENABLE| veröffentlicht), setzen Sie das Häkchen und bestätigen Sie mit \emph{Set}. Eigene Segmentfarben werden vom Kombiinstrument abgelehnt, solange diese Option deaktiviert ist.

\section{Benutzerdefinierte Farbschemata}
Der Reiter \emph{Colors} stellt einen segmentbasierten Editor für die WS2812-Hintergrundbeleuchtung bereit. Jede Zeile beschreibt ein Bereichsende, den zugehörigen Funktionsblock sowie die Farbe bzw. die Herkunft einer Basisfarbe.

\begin{enumerate}
    \item Laden Sie zunächst das aktive Mapping mit \emph{Load Scheme}. Über \emph{Add Segment} oder die Inline-Schaltflächen „+\textuparrow{}“ und „+\textdownarrow{}“ fügen Sie neue Bereiche ein. Die Dropdown-Menüs wählen die Instrumentenfunktion; mit der Basisfarb-Auswahl können Sie Haupt- oder Hintergrundfarben wiederverwenden, statt feste RGB-Werte zu setzen.
    \item Öffnen Sie den Farbwahldialog, um RGB-Töne für Segmente mit der Einstellung „Custom“ präzise zu definieren. Die Editoransicht zeigt die Komponentenwerte in Echtzeit an.
    \item Nutzen Sie die Pfeile zum Umordnen, um die physische LED-Reihenfolge abzubilden (die Segmente müssen aufsteigend sortiert bleiben). Entfernen Sie überflüssige Zeilen mit „\texttimes{}“.
    \item Sobald die Tabelle das gewünschte Layout widerspiegelt, klicken Sie auf \emph{Set Scheme}. Der Browser überträgt anschließend jedes Segment an das Kombiinstrument.
    \item Wechseln Sie unmittelbar zum Reiter \emph{Control} und drücken Sie \emph{Save parameters}. Dieser Schritt ist zwingend erforderlich\textemdash{}die Firmware hält die Segmente nur im RAM und verwirft sie nach einem Neustart, wenn sie nicht gespeichert werden.
    \item Optional können Sie die JSON-Beschreibung über \emph{Export to File} sichern oder mit \emph{Import from File} eine gespeicherte Datei wieder einspielen. \emph{Reset Scheme} stellt nach Bestätigung das Werkslayout wieder her.
\end{enumerate}

Deaktivieren Sie die Option für benutzerdefinierte Farbschemata später wieder im Reiter \emph{Settings}, fällt das Kombiinstrument automatisch in den klassischen Einfarbenmodus zurück, der von \verb|PARAMETER_MAINCOLOR_R/G/B| gesteuert wird.
