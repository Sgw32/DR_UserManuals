\chapter{Typische Konfigurationsszenarien für die \ReplicaGenOne{}}\label{ch:replica-scenarios}

Bevor Sie mit der Fehlersuche beginnen, stellen Sie sicher, dass es sich um ein klassisches \ReplicaGenOne{}-Kombiinstrument handelt (\autoref{ch:replica-setup}). \ReplicaNextLong{}-Modelle nutzen ein Wi-Fi-Portal und werden in \autoref{ch:replica-next-scenarios} behandelt.

\begin{description}
    \item[Bluetooth-Modul wird nicht gefunden] Koppeln Sie sich mit der Bluetooth-Classic-Schnittstelle des Kombiinstruments (sie sendet üblicherweise unter \texttt{Digifiz}). Die Android-App Serial Bluetooth Terminal bleibt das empfohlene Werkzeug: Stellen Sie das Zeilenendezeichen auf LF ein und vermeiden Sie reine BLE-Scanner, die das Modul nicht erkennen.
    \item[iPhone oder iPad kann keine Verbindung herstellen] \ReplicaGenOneShort{}-Kombiinstrumente verwenden Bluetooth~2.0 und sind nicht mit iOS-Geräten kompatibel. Nutzen Sie stattdessen ein Android-Telefon oder einen Computer mit serieller Bluetooth-Software.
    \item[Befehle werden in Firmware ab 2024 ignoriert] Entsperren Sie den Befehlparser mit \verb|234 123| und wiederholen Sie anschließend die gewünschte Sequenz. Speichern Sie Schnellzugriffstasten in Serial Bluetooth Terminal für häufig benötigte Werte.
    \item[Geschwindigkeitsanzeige zu hoch oder zu niedrig] Verbinden Sie sich über Serial Bluetooth Terminal, fahren Sie laut Anzeige \SI{100}{\kilo\metre\per\hour} und notieren Sie die GPS-Geschwindigkeit. Senden Sie \verb|1 <gps_value>| (z.~B. \verb|1 85|), damit \paramname{PARAMETER\_SPEEDCOEFFICIENT} der verifizierten GPS-Geschwindigkeit entspricht.
    \item[Drehzahlanzeige stimmt nicht] Firmware-Versionen vor 2024 erwarten \verb|0 <value>|, aktuelle Releases nutzen \verb|22 <value>|. Audi-Motoren benötigen meist \verb|22 3000|; halbieren oder verdoppeln Sie den Wert (z.~B. \verb|22 1500| oder \verb|22 6000|), bis die Anzeige mit dem Drehzahlmesser übereinstimmt.
    \item[Helligkeit erhöhen] Deaktivieren Sie die Automatik mit \verb|13 0| und erhöhen Sie den manuellen Wert mit \verb|14 <value>|. Werte zwischen 45 und 55 hellen das Display deutlich auf; vermeiden Sie Werte über 60, um die LED-Lebensdauer zu erhalten. Aktivieren Sie die Fotodiode anschließend wieder mit \verb|13 1|.
    \item[Uhr einstellen] Senden Sie im Serial Bluetooth Terminal \verb|255 <hours>| gefolgt von \verb|254 <minutes>|. Beispiele: \verb|255 23|, \verb|254 55| setzt 23:55; \verb|255 14|, \verb|254 30| setzt 14:30; \verb|255 2|, \verb|254 28| setzt 02:28.
    \item[Probleme mit der Tankanzeige] Klemmen Sie vor Messungen die Fahrzeugbatterie ab.
        \begin{itemize}
            \item Wandert die Anzeige von 60 auf 0, messen Sie den Widerstand des Gebers zwischen Kabelbaumpin und Masse; gültige Werte liegen typischerweise zwischen \SI{30}{} und \SI{300}{\ohm}. Reinigen Sie den Stecker und stellen Sie sicher, dass das Signal die Hauptplatine erreicht.
            \item Steht die Anzeige dauerhaft auf „voll“, suchen Sie nach einem Kurzschluss gegen Masse unter \SI{5}{\ohm} in der Geberleitung und beheben Sie ihn.
            \item Ändert sich der Wert nie, vergleichen Sie den Geberwiderstand bei vollem und leerem Tank. Tauschen Sie den Sensor, wenn der Widerstand unverändert bleibt.
        \end{itemize}
    \item[Kraftstoffdurchfluss wirkt unplausibel] Der Durchflusskanal wird emuliert, sofern kein Saugrohrdrucksensor verbaut ist. Betrachten Sie die Anzeige daher als Richtwert.
    \item[Kühlmittelanzeige ungenau] Passen Sie \paramname{PARAMETER\_COOLANT\_MIN\_R} und \paramname{PARAMETER\_COOLANT\_MAX\_R} an. Beispiel: \verb|27 30| verkürzt die Skala, sodass die Markierung „1~bar“ etwa \SI{30}{\celsius} entspricht.
    \item[Öl- oder Außentemperatur fehlen] Ein Wert von \texttt{-999} oder eine feste Anzeige weist auf ein Sensorproblem hin. Messen Sie bei abgeklemmter Batterie und kaltem Motor den Sensorwiderstand zwischen Kabelbaumpin und Masse. Ölsensoren sollten etwa \SI{2}{\kilo\ohm} \ensuremath{\pm}\SI{0.3}{\kilo\ohm} liefern; Außensensoren etwa \SI{10}{\kilo\ohm} \ensuremath{\pm}\SI{2}{\kilo\ohm}. Korrigieren Sie die Anzeige bei Bedarf über \paramname{PARAMETER\_NORMAL\_RESISTANCE\_OIL} (Befehl~20) oder \paramname{PARAMETER\_NORMAL\_RESISTANCE\_AMB} (Befehl~21). Anhaltende Fehler sollten mit \verb|adc 0|-Protokollen dokumentiert und an den Support von PHOL-LABS Kft gemeldet werden.
\end{description}

Erfassen Sie rohe Sensordaten mit \verb|adc 0|, falls das Problem weiter besteht, und übermitteln Sie die Ergebnisse zur Analyse an die Entwickler des Kombiinstruments.
